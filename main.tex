\documentclass[floatsintext,man]{apa6}

\usepackage{amssymb,amsmath}
\usepackage{ifxetex,ifluatex}
\usepackage{fixltx2e} % provides \textsubscript
\ifnum 0\ifxetex 1\fi\ifluatex 1\fi=0 % if pdftex
  \usepackage[T1]{fontenc}
  \usepackage[utf8]{inputenc}
\else % if luatex or xelatex
  \ifxetex
    \usepackage{mathspec}
    \usepackage{xltxtra,xunicode}
  \else
    \usepackage{fontspec}
  \fi
  \defaultfontfeatures{Mapping=tex-text,Scale=MatchLowercase}
  \newcommand{\euro}{€}
\fi
% use upquote if available, for straight quotes in verbatim environments
\IfFileExists{upquote.sty}{\usepackage{upquote}}{}
% use microtype if available
\IfFileExists{microtype.sty}{\usepackage{microtype}}{}

% Table formatting
\usepackage{longtable, booktabs}
\usepackage{lscape}
% \usepackage[counterclockwise]{rotating}   % Landscape page setup for large tables
\usepackage{multirow}		% Table styling
\usepackage{tabularx}		% Control Column width
\usepackage[flushleft]{threeparttable}	% Allows for three part tables with a specified notes section
\usepackage{threeparttablex}            % Lets threeparttable work with longtable

% Create new environments so endfloat can handle them
\newenvironment{ltable}
  {\begin{landscape}\begin{center}\begin{threeparttable}}
  {\end{threeparttable}\end{center}\end{landscape}}

\newenvironment{lltable}
  {\begin{landscape}\begin{center}\begin{ThreePartTable}}
  {\end{ThreePartTable}\end{center}\end{landscape}}

\usepackage{ifthen} % Only add declarations when endfloat package is loaded
\ifthenelse{\equal{man}{\string jou}}{%
  \DeclareDelayedFloatFlavor{ThreePartTable}{table} % Make endfloat play with longtable
  \DeclareDelayedFloatFlavor{ltable}{table} % Make endfloat play with lscape
  \DeclareDelayedFloatFlavor{lltable}{table} % Make endfloat play with lscape & longtable
}{}%


% The following enables adjusting longtable caption width to table width
% Solution found at http://golatex.de/longtable-mit-caption-so-breit-wie-die-tabelle-t15767.html
\makeatletter
\newcommand\LastLTentrywidth{1em}
\newlength\longtablewidth
\setlength{\longtablewidth}{1in}
\newcommand\getlongtablewidth{%
 \begingroup
  \ifcsname LT@\roman{LT@tables}\endcsname
  \global\longtablewidth=0pt
  \renewcommand\LT@entry[2]{\global\advance\longtablewidth by ##2\relax\gdef\LastLTentrywidth{##2}}%
  \@nameuse{LT@\roman{LT@tables}}%
  \fi
\endgroup}


  \usepackage{graphicx}
  \makeatletter
  \def\maxwidth{\ifdim\Gin@nat@width>\linewidth\linewidth\else\Gin@nat@width\fi}
  \def\maxheight{\ifdim\Gin@nat@height>\textheight\textheight\else\Gin@nat@height\fi}
  \makeatother
  % Scale images if necessary, so that they will not overflow the page
  % margins by default, and it is still possible to overwrite the defaults
  % using explicit options in \includegraphics[width, height, ...]{}
  \setkeys{Gin}{width=\maxwidth,height=\maxheight,keepaspectratio}
\ifxetex
  \usepackage[setpagesize=false, % page size defined by xetex
              unicode=false, % unicode breaks when used with xetex
              xetex]{hyperref}
\else
  \usepackage[unicode=true]{hyperref}
\fi
\hypersetup{breaklinks=true,
            pdfauthor={},
            pdftitle={Assumptions of the process-dissociation procedure are violated in sequence learning},
            colorlinks=true,
            citecolor=blue,
            urlcolor=blue,
            linkcolor=black,
            pdfborder={0 0 0}}
\urlstyle{same}  % don't use monospace font for urls

\setlength{\parindent}{0pt}
%\setlength{\parskip}{0pt plus 0pt minus 0pt}

\setlength{\emergencystretch}{3em}  % prevent overfull lines

\setcounter{secnumdepth}{0}

% Manuscript styling
\captionsetup{font=singlespacing,justification=justified}
\usepackage{csquotes}
\usepackage{upgreek}



\usepackage{tikz} % Variable definition to generate author note

% fix for \tightlist problem in pandoc 1.14
\providecommand{\tightlist}{%
  \setlength{\itemsep}{0pt}\setlength{\parskip}{0pt}}

% Essential manuscript parts
  \title{Assumptions of the process-dissociation procedure are violated in
sequence learning}

  \shorttitle{Invariance assumption in sequence learning}


  \author{Marius Barth, Christoph Stahl, \& Hilde Haider}

  \def\affdep{{"", "", ""}}%
  \def\affcity{{"", "", ""}}%

  \affiliation{
    \vspace{0.5cm}
          \textsuperscript{} University of Cologne  }

  \authornote{
    \newcounter{author}
    Marius Barth, Christoph Stahl, Hilde Haider, Department of Psychology,
    University of Cologne.
    
    CS and HH designed the research; MB, CS, and HH planned the studies; MB
    ran the studies; MB and CS analyzed the data; MB, CS, and HH wrote the
    paper. We thank Jan Czarnomski, Conni Lebbing, Friederike Neugebauer,
    and Imge Ürer for help with data collection. We also thank Ute J. Bayen
    for generously opening her laboratory for data collection.
    
    This work was funded by Deutsche Forschungsgemeinschaft grant
    STA-1269/1-1 to CS and grant HA-5447/8-1 to HH.
    
    Data, code, and materials necessary to reproduce the analyses reported
    in this article are available at \url{https://github.com/methexp/pdl2}.

                      Correspondence concerning this article should be addressed to Marius Barth, Herbert-Lewin-Str. 2, D-50931 Köln, Germany. E-mail: \href{mailto:marius.barth@uni-koeln.de}{\nolinkurl{marius.barth@uni-koeln.de}}
                                    }

  \note{\emph{(Unpublished manuscript, 2016-10-25)}.}

  \abstract{In sequence learning, a process-dissociation (PD) approach has been
proposed to dissociate implicit and explicit learning processes. Applied
to the popular generation task, participants perform two different task
versions: an \emph{inclusion} condition asking them to re-generate the
learned sequence, and an \emph{exclusion} condition asking them to avoid
generating the learned sequence. Whereas accurate performance under
inclusion may be based on either implicit and/or explicit knowledge,
avoiding to generate the learned sequence requires controllable explicit
sequence knowledge. The PD approach yields separate estimates of
explicit and implicit knowledge that are derived from the same task and
therefore avoids many problems of previous measurement approaches.
However, the PD approach rests on the critical assumption that the
implicit and explicit processes are invariant across inclusion and
exclusion conditions. We tested whether the invariance assumptions holds
for the PD generation task. Across three studies using first-order as
well as second-order regularities, invariance of the controlled process
was found to be violated. In particular, despite extensive amounts of
practice, explicit knowledge was not exhaustively expressed in the
exclusion condition. We discuss the implications of these findings for
the use of process-dissociation in assessing implicit knowledge.}
  \keywords{sequence learning, process-dissociation procedure, invariance assumption \\

    \indent Word count: 14,041
  }




\begin{document}

\maketitle



\section{Introduction}\label{introduction}

Riding a bicycle is an easy task, but most of us will be hard-pressed to
describe in detail the movements necessary for pedaling, keeping
direction, and maintaining balance. Capturing this intuition, theories
of human learning distinguish two types of knowledge: Explicit knowledge
is acquired when a learner becomes aware of an environmental regularity
and stores it in episodic memory; in contrast, implicit knowledge
reflects regularities in the environment that may have been acquired
without becoming aware of them (Shanks \& St. John, 1994). A classical
paradigm for the study of implicit learning, the serial reaction time
task (SRTT; Nissen \& Bullemer, 1987) has participants respond to
stimuli presented at four horizontal screen locations by pressing the
key that corresponds to the stimulus location. Unbeknownst to
participants, the stimulus locations follow a regular sequence. With
practice, participants learn to respond faster on trials with regular
stimulus-location transitions than on irregular transitions. Despite
this performance advantage for responses that follow the sequence,
participants are often unable to verbalize any knowledge about the
sequence.

The measurement of implicit and explicit knowledge in this and other
paradigms has been subject of debate. In the domain of sequence
learning, the RT advantage of regular over irregular transitions has
often been taken to indicate implicit knowledge. To assess explicit
knowledge, verbal reports have been used, but they have been criticized
as insensitive and potentially distorted by conservative reporting
criteria, and they also differ from the RT measure in reliability (i.e.,
verbal reports rely on only a single data point per participant), and in
immediacy (i.e., verbal reports are assessed only after the SRTT) (see
Shanks \& St. John, 1994). The application of the process-dissociation
(PD) approach (introduced by Jacoby, 1991, to dissociate implicit and
explicit memory) to sequence learning has therefore been an important
improvement (Buchner, Steffens, Rothkegel, \& Erdfelder, 1997; Curran,
2001; Destrebecqz \& Cleeremans, 2001). In the PD approach, measures of
implicit and explicit knowledge are derived from two variants of the
same task, thereby largely eliminating the criticized confounds.
Specifically, under \emph{inclusion} instructions participants are asked
to apply their explicit knowledge when solving the task, so that correct
inclusion performance can arise from both explicit and implicit
knowledge; in contrast, under \emph{exclusion} instructions they are
asked to refrain from using explicit knowledge, so that correct
exclusion performance can be attributed only to implicit knowledge.

\subsection{Process dissociation in the generation
task}\label{process-dissociation-in-the-generation-task}

Destrebecqz and Cleeremans (2001) applied the PD approach to the
generation task. Participants were instructed, after finishing the SRTT,
to generate a sequence that is either similar (in the inclusion
condition) or dissimilar (in the exclusion condition) to that observed
during the SRTT. To the degree that participants can generate a similar
sequence under the inclusion instruction, they can be said to have
acquired knowledge about the sequence; yet, this knowledge may reflect
implicit and/or explicit knowledge because both may be used to
re-generate the learned sequence. However, only explicit knowledge is
assumed to be under participants' control: When asked to generate a
sequence that is dissimilar to the learned sequence -- that is, to
\emph{exclude} their explicit knowledge -- participants can avoid
generating similar transitions only \emph{to the degree that their
sequence knowledge is explicit}. To the degree that their sequence
knowledge is implicit, they would still generate a sequence
\emph{similar} to the learned sequence despite being instructed to do
the opposite. Based on this logic, conclusions about the presence or
absence of explicit knowledge can be drawn from performance differences
between the inclusion and exclusion conditions; conclusions about the
presence or absence of implicit knowledge can be drawn from differences
between exclusion performance and a control condition or chance
baseline.

The PD generation task has repeatedly been used to investigate sequence
learning (e.g., Destrebecqz \& Cleeremans, 2001, 2003; Q. Fu, Dienes, \&
Fu, 2010; Q. Fu, Fu, \& Dienes, 2008; Haider, Eichler, \& Lange, 2011;
Jiménez, Vaquero, \& Lupiáñez, 2006; Mong, McCabe, \& Clegg, 2012;
Norman, Price, \& Duff, 2006; Shanks, Rowland, \& Ranger, 2005;
Wilkinson \& Shanks, 2004), and results showed some convergent validity:
Participants who had acquired explicit knowledge -- as measured by the
PD procedure -- were able to drastically reduce their RT during the
learning phase by actively predicting the next response; this was not
the case for participants who did not show evidence for explicit
knowledge in the PD task (Haider et al., 2011). Investigations focusing
on the relative contributions of implicit and explicit knowledge in
sequence learning have, however, yielded mixed results: While some
studies found evidence for implicit but no explicit knowledge, or
reported combinations of implicit and explicit knowledge acting together
(e.g., Destrebecqz \& Cleeremans, 2001, 2003; Q. Fu et al., 2010, 2008;
Haider et al., 2011; Jiménez et al., 2006; Mong et al., 2012; Norman et
al., 2006), other studies found only evidence for explicit knowledge
(Shanks et al., 2005; e.g., Wilkinson \& Shanks, 2004). Whereas it is
possible that moderating variables (e.g., the response-stimulus
interval, cued vs.~uncued generation task) may be identified that can
account for these discrepancies, they may also (at least in part) arise
from unwarranted assumptions of the PD method as discussed next.

\subsubsection{Process dissociation and its
assumptions}\label{process-dissociation-and-its-assumptions}

The PD approach has become a popular and versatile tool for measuring
the relative contributions of implicit or automatic versus explicit or
controlled processes in a variety of tasks (Yonelinas \& Jacoby, 2012).
It can be formalized as a set of equations describing inclusion (\(I\))
and exclusion (\(E\)) performance as a function of the probabilities of
controlled process, \(C\), and the automatic process, \(A\), as follows:
\[I=C+(1-C)*A\] and \[E=(1-C)*A\] These equations express the notions
that (1) correct responses under inclusion can arise from either the
controlled process (with probability \(C\)) or, given that it fails
(with probability \(1-C\)), from the automatic process \(A\); and (2)
correct responses under exclusion are solely due to the automatic
process in the absence of the influence of the controlled process,
\((1-C)*A\). Solving these equations for \(C\) and \(A\) (or using
parameter estimation techniques for multinomial models) yields estimates
of the contributions of the controlled and automatic process.

The validity of the PD method and model has been the target of debate
since its introduction by Jacoby (1991; see, e.g., Buchner, Erdfelder,
\& Vaterrodt-Plünnecke, 1995; Curran \& Hintzman, 1995). This is because
the PD approach is not a theory-free measurement tool but rests on a set
of strong and possibly problematic assumptions. First and obviously, it
assumes the existence of two qualitatively different---controlled and
automatic---processes, and it aims to measure the magnitude of their
respective contributions. It is, however, not well-suited for comparing
single- and dual-process models: To illustrate, Ratcliff, Van Zandt and
McKoon (1995) found that data generated from a single-process model
could produce a data pattern that, when analyzed using the PD approach,
appears to support the existence -- and differential contributions -- of
two qualitatively distinct processes. This implies that empirical
dissociations between the controlled and automatic estimates do not
necessarily imply the existence of two qualitatively different
underlying processes.

Second, it is assumed that both processes operate independently; that
is, on each trial, both the explicit and the implicit process attempt to
produce a candidate response in parallel, and their respective candidate
responses are not influencing each other. In particular, the response
produced by the automatic process is assumed to be uninfluenced by
whether the controlled process produces the same or a different
response. Relatedly, the model assumes that independence holds across
persons and items; when data are aggregated over (potentially
heterogeneous) participants and items, a violation can lead to biases in
parameter estimates. There has been considerable debate about the
independence assumption in applications of the PD to episodic memory
paradigms (Curran \& Hintzman, 1995, 1997; Hintzman \& Curran, 1997;
Jacoby \& Shrout, 1997). Evidence suggests that aggregation independence
may often be violated; hierarchical extension of the PD model have been
proposed to address this problem (Rouder, Lu, Morey, Sun, \& Speckman,
2008).

Third, it is assumed that both the controlled and automatic processes
are \emph{invariant} across the inclusion and exclusion instructions.
This is reflected in the PD equations by the use of a single parameter
\(C\) instead of separate parameters for inclusion and exclusion; in
other words, the PD equations represent a simplified model that
incorporates the invariance assumption
\(C = C_{Inclusion}=C_{Exclusion}\). Similarly, the PD equations include
only a single parameter \(A\), reflecting the simplifying assumption
that the automatic process is invariant across inclusion and exclusion,
\(A = A_{Inclusion} = A_{Exclusion}\). If the PD instruction affects
those cognitive processes, the PD equations do no longer yield valid
estimates. Recently, the invariance assumption was indeed found to be
violated for the controlled process in three different paradigms
(Klauer, Dittrich, Scholtes, \& Voss, 2015). The goal of the present
study is to test whether the PD model's invariance assumption holds for
the generation task.

\subsection{Invariance assumption: Consequences of
violations}\label{invariance-assumption-consequences-of-violations}

Violations of the invariance assumptions may considerably distort
parameter estimates and substantive conclusions (e.g., Buchner et al.,
1995; Klauer et al., 2015). This is also true for the generation task.
Assume first that participants have explicit but no implicit sequence
knowledge: In the inclusion task, participants would easily express
their knowledge. In the exclusion task, they would strategically
generate a different sequence to avoid regular responses in this task.
However, they would not notice that this different sequence also
contains transitions of the (to-be-avoided) sequence. If this were the
case, then explicit knowledge would be more likely to be expressed in
the inclusion task than in the exclusion task. This implies that
explicit knowledge would successfully lead to improved inclusion
performance, but would fail to adequately reduce the rate of regular
transitions generated under exclusion instructions. However, because the
controlled parameter is assumed to be equal across inclusion and
exclusion, this effect would distort estimates of the other parameters;
in particular, it would lead to inflated estimates of implicit
knowledge. In this case, researchers would erroneously conclude that
both explicit and implicit knowledge were present.

Next, assume that participants have acquired implicit but no explicit
sequence knowledge. Assume further that they adopt a liberal criterion
under the inclusion instruction, allowing them to correctly reproduce a
substantial proportion of regular transitions by relying on motor
fluency. Under exclusion instructions, lacking explicit and therefore
controllable knowledge, they might try to reduce the generation rate of
regular transitions by adopting one of several response strategies
(e.g., subjective randomness, persevering specific patterns; Stahl,
Barth, \& Haider, 2015). As a consequence, the acquired implicit
knowledge would be expressed to a greater degree under inclusion than
under exclusion instructions, a pattern that would typically be
interpreted as the presence of explicit knowledge. Taken together,
conclusions about the presence or absence of implicit and explicit
knowledge, as well as their relative contributions across conditions,
may be erroneous if one or both of the invariance assumptions are
violated. It is therefore important to test whether these assumptions
can be upheld in the investigation of sequence knowledge using the PD
approach to the generation task.

\section{Overview of present studies}\label{overview-of-present-studies}

The present study aimed at testing, in the PD generation task, the
invariance assumption for automatic and controlled processes. For this
purpose, it was necessary to extend the traditional PD design by
orthogonally manipulating explicit and implicit knowledge (see also
Klauer et al., 2015). We manipulated \emph{explicit} knowledge by
explicitly informing participants, after the SRTT training phase, about
a subset of the regular transitions (e.g., 1 out of 6) of the sequence.
By presenting information about the transitions \emph{after training} we
ensured that participants did not use that information during the SRTT
to strategically search for more regular transitions (i.e., we made sure
the manipulation did not affect the amount of sequence knowledge
acquired during training). We manipulated \emph{implicit} knowledge by
varying the amount of regularity present in the SRTT training sequence.
For this purpose, we used materials with a mere probabilistic
regularity; such materials typically produce robust implicit knowledge
in the absence of explicit knowledge.

We then fit an extended process-dissociation model \(\mathcal{M}_1\)
that allowed for testing the invariance assumption of both the
controlled and the automatic process: The model provided us with
separate estimates for these processes for both inclusion and exclusion
tasks; and we used the differences between these estimates to test the
invariance assumption. This model relies on the auxiliary assumptions
that each experimental manipulation selectively influenced only one of
both processes; these assumptions are tested by goodness-of-fit tests
proposed by Klauer (2010). Moreover, in order to justify the auxiliary
assumptions, we specified a standard process-dissociation model
\(\mathcal{M}_2\) that does not enforce the auxiliary assumptions but
enforces the invariance assumption; model comparison techniques (DIC;
Spiegelhalter, Best, Carlin, \& Van Der Linde, 2002) were then used to
compare model \(\mathcal{M}_1\) and model \(\mathcal{M}_2\). If model
\(\mathcal{M}_1\) is favored over model \(\mathcal{M}_2\), this can be
taken as evidence in favor of our auxiliary assumptions over the
invariance assumption. Finally, instead of aggregating data, we used
hierarchical Bayesian extensions of all models (e.g., Klauer, 2010;
Rouder \& Lu, 2005; Rouder et al., 2008).

The outline of this article is as follows: In Experiment 1, we applied
the just-presented method to an SRTT with first-order conditional
material. In Experiment 2, we replicated our findings from Experiment 1
and extended them to second-order conditional material. Finally, because
Experiments 1 and 2 found a violation of invariance for the controlled
process, Experiment 3 explored potential mechanisms underlying this
violation of invariance. We discuss the implications of our findings for
the validity of the PD model equations as well as ordinal
interpretations of findings obtained with the PD procedure as applied to
the generation task. Furthermore, we point out directions in which the
generation task could be developed to provide an improved measure of
implicit and explicit knowledge in sequence learning.

\section{Experiment 1}\label{experiment-1}

Experiment 1 tested the invariance assumption for automatic and
controlled processes using materials with first-order regularity. We
implemented two different levels of implicit knowledge by presenting
either random or probabilistic sequences to participants during the SRT
task. Orthogonally, we implemented two different levels of explicit
knowledge by experimentally inducing such knowledge: After the SRT task,
we informed one half of participants about one of the six transitions in
the sequence.

\subsection{Method}\label{method}

\subsubsection{Design}\label{design}

The study realized a 2 (\emph{material}: random vs.~probabilistic)
\(\times\) 2 (\emph{explicit knowledge}: no transition revealed vs.~one
transition revealed) \(\times\) 2 (\emph{PD instruction}: inclusion
vs.~exclusion) \(\times\) 2 (\emph{block order}: inclusion first
vs.~exclusion first) design with repeated measures on the \emph{PD
instruction} factor.

\subsubsection{Participants}\label{participants}

One hundred and twenty-one participants (87 women) aged between 17 and
51 years (\(M = 23.7\) years) completed the study. Most were
undergraduates from University of Cologne. Participants were randomly
assigned to experimental conditions. They received either course credit
or 3.50 Euro for their participation.

\subsubsection{Materials}\label{materials}

We used two different types of material:

\begin{itemize}
\tightlist
\item
  A \emph{random} sequence was randomly generated for each participant
  anew by drawing with replacement from a uniform distribution of six
  response locations.
\item
  A \emph{probabilistic} sequence was generated from the first-order
  conditional sequence \(2-6-5-3-4-1\). With a probability of \(.6\), a
  stimulus location was followed by the next location from this
  sequence; otherwise, another stimulus location was randomly chosen
  from a uniform distribution.
\end{itemize}

In both materials there were no direct repetitions of response
locations. In the random group, there was no \enquote{correct} sequence,
and transition frequencies varied across persons. To compute the
dependent variable in the generation task (i.e., the proportion of
rule-adhering or regular transitions), we used the generating sequence
for participants who worked on \emph{probabilistic} material; for
participants who worked on \emph{random} material, we determined an
individual criterion for each participant based on their individual
transition frequencies during learning: For each participant, the
sequence that best fitted the transitions observed by that participant
during the \emph{acquisition} phase served as a criterion for the
\emph{generation} phase. For the group that was instructed about a
regular transition, this \emph{criterion sequence} also contained the
revealed transition.

\subsubsection{Procedure}\label{procedure}

The experiment consisted of three consecutive parts: Participants first
worked on a SRTT (the \emph{acquisition task}), followed by a
\emph{generation task} and, finally, a debriefing phase. In the
acquisition task, participants performed a SRTT consisting of 8 blocks
with 144 trials each (for a total of 1,152 responses). SRTT and
generation task were run on 17" CRT monitors (with a screen resolution
of \(1{,}024~\text{px} \times 768~\text{px}\)). The viewing distance was
approximately \(60~\text{cm}\). A horizontal sequence of six white
squares (\(56~\text{px}\)) was presented on a gray screen. The distance
between squares was \(112~\text{px}\). Each screen location corresponded
to a key on a QWERTZ keyboard (from left to right Y, X, C, B, N, M).
Participants had to respond whenever a square's color changed from white
to red by pressing the corresponding key. They were instructed to place
the left ring-, middle- and index fingers on the keys Y, X and C. The
right index-, middle- and ring fingers were to be placed on keys B, N
and M. There was no time limit for responses in the learning phase (nor
in the generation phase). A warning beep indicated an incorrect
response. The response-stimulus interval (RSI) was \(250~\text{ms}\).

Following the SRTT phase, participants were told that stimulus locations
during the SRTT followed an underlying sequential structure (but were
not informed about the exact sequence). They were then asked to try to
generate a short sequence of six locations that followed this structure.

Before working on practice blocks, one transition was revealed to one
half of the participants. They were told to memorize that transition and
to use this knowledge in the following tasks.

The generation task contained a counterbalanced order of inclusion
versus exclusion blocks. Under inclusion (exclusion) instructions,
participants were told to generate a sequence as similar (dissimilar) as
possible to the sequence from the acquisition task. For both task,
participants were instructed to follow their intuition if they had no
explicit knowledge about the underlying sequence. Participants who had
received information about a transition were instructed to include
(exclude) the revealed transition.

To familiarize participants with both inclusion and exclusion
instructions, they worked on short practice blocks of twelve consecutive
responses. Prior to the inclusion task, two practice blocks involved
inclusion instructions; prior to the exclusion task, the first practice
block was performed under inclusion instructions and the second practice
block was performed under exclusion instructions. If participants who
were explicitly informed about one transition failed to include
(exclude) the revealed transition in practice blocks, they were informed
that they did something wrong; the already revealed transition was again
presented and two additional practice blocks had to be performed. This
procedure was repeated until the revealed transition was successfully
included (excluded) in two consecutive practice blocks. In the main
block of the generation task, participants freely generated 120
consecutive response locations. Question marks appeared at all locations
and participants' key presses were reflected by the corresponding
square's color changing to red. Direct repetitions were explicitly
discouraged and were followed by a warning beep.

Upon completing the computerized task, participants were asked to
complete a questionnaire containing the following items (translated from
German): (1) \enquote{One of the tasks mentioned a sequence in which the
squares lit up during the first part of the study. In one of the
experimental conditions, the squares did indeed follow a specific
sequence. Do you think you were in this condition or not?}, (2)
\enquote{How confident are you (in \%)?}, and (3) \enquote{Can you
describe the sequence in detail?}. Subsequently, participants were asked
to indicate, for each of the six response keys, the next key in the
sequence on a printed keyboard layout and to indicate how confident they
were in this decision. Finally, participants were thanked and debriefed.

\subsubsection{Data analysis}\label{data-analysis}

All analyses were performed using the R software (R Core Team, 2016) and
Stan (Carpenter et al., in press). For the model-based analyses, we used
hierarchical Bayesian extensions of the process-dissociaton model
(Klauer, 2010; Rouder \& Lu, 2005; Rouder et al., 2008). The first level
of this hierarchical model extended the traditional process-dissociation
model by allowing for a violation of the invariance assumption: The
controlled and automatic processes were allowed to vary as a function of
instruction (inclusion vs.~exclusion), \[
\begin{aligned}
  I_{ij} & =  C_{ijm} + (1-C_{ijm}) A_{ijm}, m = 1\\
  E_{ij} & =  (1-C_{ijm}) A_{ijm}, m = 2
\end{aligned}
\] where \(i\) indexes participants, \(j\) indexes transition type
(i.e., revealed: \(j = 1\); nonrevealed: \(j = 2\)), and \(m\) indexes
the \emph{PD instruction} condition (inclusion: \(m=1\); exclusion:
\(m=2\)).

Parameters \(C_{ijm}\) and \(A_{ijm}\) are probabilities in the range
between zero and one; following previous work (e.g. Albert \& Chib,
1993; Klauer et al., 2015; Rouder et al., 2008), we used a probit
function to link these probabilities to the second-level parameters as
follows:

\[
  C_{ijm} = \begin{cases}
    \Phi(\mu_{km}^{(C)} + \delta_{im}^{(C)}) & \text{if } j=1 \text{ (item has been revealed)}\\
                                            0 & \text{if } j=2 \text{ (item has not been revealed)}\\
    \end{cases}
\] and \[
  A_{ijm} = \Phi(\mu_{jkm}^{(A)} + \delta_{ijm}^{(A)})
\]

where \(\Phi\) denotes the standard normal cumulative distribution
function, \(\mu_{km}^{(C)}\) is the fixed effect of material \(k\) (that
participant \(i\) worked on during the SRTT) and \emph{PD instruction}
condition \(m\) on controlled processes. \(\delta_{im}^{(C)}\) is the
\(i\)th participant's deviation from his or her group's mean.

Accordingly, \(\mu_{jkm}^{(A)}\) is the fixed effect of transition type
\(j\), material \(k\), and \emph{PD instruction} condition \(m\) on
automatic processes, and \(\delta_{ijm}^{(A)}\) is the \(i\)th
participant's deviation from the corresponding mean. Priors on
parameters are given in the Appendix.

Note that this specification imposes two auxiliary assumptions to the
model: First, it is assumed that controlled processes \(C\) are set to
zero for nonrevealed transitions (i.e., \(C=0\) for \(j=2\)), in other
words, we assumed that no explicit knowledge has been acquired during
the SRT phase. Second, it is assumed that automatic processes \(A\) do
not vary as a function of the between-subjects manipulation of explicit
knowledge \(l\) (i.e., \(A_{l=1} = A_{l=2}\)). These assumptions allowed
us to relax and test the invariance assumption by obtaining separate
estimates of both \(C\) and \(A\) for the inclusion and exclusion
conditions (note that a \emph{full} model relaxing all three assumptions
cannot be estimated).

To assess goodness of fit, we used posterior predictive model checks as
proposed by Klauer (2010): Statistic \(T_{A1}\) summarizes how well the
model describes the individual category counts for the eight categories
(revealed vs.~nonrevealed transitions \(\times\) correct vs.~incorrect
\(\times\) inclusion vs.~exclusion). Statistic \(T_{B1}\) summarizes how
well the model describes the covariations in the data across
participants.

Additionally, we also estimated a model \(\mathcal{M}_2\) that does not
impose the auxiliary assumptions but enforces the invariance assumptions
(i.e., parameters were not allowed to vary as a function of PD
instruction condition \(m\)):

\[
\begin{aligned}
  I_{ij} & =  C_{ij} + (1-C_{ij}) A_{ij}\\
  E_{ij} & =  (1-C_{ij}) A_{ij}
\end{aligned}
\]

The second-level equations of model \(\mathcal{M}_2\) are then given by:

\[
  C_{ij} = \Phi(\mu_{jkl}^{(C)} + \delta_{ij}^{(C)})
\] and \[
  A_{ij} = \Phi(\mu_{jkl}^{(A)} + \delta_{ij}^{(A)})
\]

where \(i\) indexes participants, \(j\) indexes transition type, \(k\)
indexes the learning material that participant \(i\) worked on during
the SRTT, and \(l\) indexes the manipulation of explicit knowledge
(i.e., whether or not a transition has been revealed to participant
\(i\)). Note that, given this model specification, separate parameters
are estimated for each between-subjects condition \(kl\) and each
transition type \(j\), while the invariance assumption is maintained
(i.e., there is no index \(m\) for \emph{PD instruction} in the model
equations).

These two models were compared using the deviance information criterion
DIC (Spiegelhalter et al., 2002); if model \(\mathcal{M}_1\) outperforms
model \(\mathcal{M}_2\), it can be concluded that the auxiliary
assumptions are less problematic than the invariance assumptions.
Furthermore, model \(\mathcal{M}_1\) yields separate estimates of
controlled and automatic processes for both inclusion and exclusion. The
invariance assumption can be targeted directly by calculating the
posterior differences \(A_{I} - A_{E}\) and \(C_{I} - C_{E}\): If the
posterior distributions of these differences include zero, it can be
concluded that the respective invariance assumption holds; if the
posterior does not contain zero, it can be concluded that the respective
invariance assumption is violated.

\subsection{Results}\label{results}

We first analyzed the performance data from the SRT task to determine
whether sequence knowledge had been acquired during the task. Next, we
analyzed generation task performance using hierarchical PD models.

\subsubsection{Acquisition task}\label{acquisition-task}

If participants acquired knowledge about the (probabilistic) regularity
underlying the sequence of key presses, we expect a performance
advantage for regular over irregular transitions, reflected in reduced
RT and/or error rate. If this advantage is due to learning, it is
expected to increase over SRTT blocks.

\paragraph{Reaction times}\label{reaction-times}



For RT analyses, we excluded the first trial of each block because the
first location cannot be predicted, as well as error trials, trials
succeeding an error, reactions faster than 50 ms and slower than 1,000
ms. Figure 1 shows reaction times during the SRTT.

We conducted a 2 (\emph{Material}: Random vs.~Probabilistic) \(\times\)
8 (\emph{Block number}) \(\times\) 2 (\emph{FOC transition status}:
regular vs.~irregular) ANOVA that revealed a main effect of
\emph{material}, \(F(1, 119) = 8.11\), \(\mathit{MSE} = 39,617.25\),
\(p = .005\), \(\eta^2_G = .055\); a main effect of \emph{block number}
\(F(4.89, 582.06) = 33.35\), \(\mathit{MSE} = 1,032.91\), \(p < .001\),
\(\eta^2_G = .029\); a main effect of \emph{FOC transition status},
\(F(1, 119) = 125.46\), \(\mathit{MSE} = 714.88\), \(p < .001\),
\(\eta^2_G = .016\); an interaction of \emph{material} and \emph{FOC
transition status}, \(F(1, 119) = 121.57\), \(\mathit{MSE} = 714.88\),
\(p < .001\), \(\eta^2_G = .015\); an interaction of \emph{block number}
and \emph{FOC transition status}, \(F(6.32, 752.52) = 10.68\),
\(\mathit{MSE} = 197.96\), \(p < .001\), \(\eta^2_G = .002\); and a
three-way interaction between \emph{material}, \emph{block number}, and
\emph{FOC transition status}, \(F(6.32, 752.52) = 5.70\),
\(\mathit{MSE} = 197.96\), \(p < .001\), \(\eta^2_G = .001\).

Separate ANOVAs for each \emph{material} condition yielded, for random
material, only a significant main effect of \emph{block number},
\(F(4.38, 258.47) = 13.09\), \(\mathit{MSE} = 1,276.78\), \(p < .001\),
\(\eta^2_G = .026\), with RTs decreasing over blocks (all other
\emph{F}s \textless{} 1). For probabilistic material, in contrast, we
obtained main effects of \emph{block number},
\(F(5.07, 304.28) = 22.09\), \(\mathit{MSE} = 891.30\), \(p < .001\),
\(\eta^2_G = .035\); and of \emph{transition status},
\(F(1, 60) = 182.32\), \(\mathit{MSE} = 976.60\), \(p < .001\),
\(\eta^2_G = .061\) (i.e.~responses to regular transitions were faster
than those for irregular transitions); importantly, we also obtained an
interaction of \emph{block number} and \emph{transition status},
\(F(5.93, 356.02) = 15.83\), \(\mathit{MSE} = 194.03\), \(p < .001\),
\(\eta^2_G = .007\), showing that the RT difference between regular and
irregular transitions increased over blocks, indicating learning of the
regularities inherent in the probabilistic material.

\paragraph{Error rates}\label{error-rates}



For analyses of error rates, we excluded the first trial of each block.
Figure 2 shows error rates during acquisition. We conducted a 2
(\emph{Material}: Random vs.~Probabilistic) \(\times\) 8 (\emph{Block
number}) \(\times\) 2 (\emph{FOC transition status}: regular
vs.~irregular) ANOVA that revealed a main effect of \emph{block number},
\(F(5.83, 693.83) = 6.06\), \(\mathit{MSE} = 11.83\), \(p < .001\),
\(\eta^2_G = .016\), indicating that error rates increased over blocks,
and a main effect of \emph{FOC transition status},
\(F(1, 119) = 38.19\), \(\mathit{MSE} = 13.49\), \(p < .001\),
\(\eta^2_G = .019\), indicating that error rates were higher for
nonregular transitions. The interaction of \emph{material} and \emph{FOC
transition status} was also significant, \(F(1, 119) = 27.61\),
\(\mathit{MSE} = 13.49\), \(p < .001\), \(\eta^2_G = .014\), reflecting
the finding that the effect of the latter factor was limited to the
probabilistic material. The three-way interaction of \emph{material},
\emph{block number}, and \emph{FOC transition status} was however not
significant, \(F(6.55, 778.97) = 1.84\), \(\mathit{MSE} = 7.94\),
\(p = .082\), \(\eta^2_G = .004\).

To disentangle these interactions, we analyzed both \emph{material}
groups separately. As for RT, an ANOVA for the random material group
revealed only a main effect of \emph{block number},
\(F(4.94, 291.45) = 2.50\), \(\mathit{MSE} = 16.03\), \(p = .031\),
\(\eta^2_G = .013\) (all other \emph{F}s \textless{} 1). The
probabilistic material group showed a main effect of \emph{block number}
\(F(5.73, 343.65) = 4.63\), \(\mathit{MSE} = 10.29\), \(p < .001\),
\(\eta^2_G = .022\), and a main effect of \emph{FOC transition status},
\(F(1, 60) = 62.50\), \(\mathit{MSE} = 14.23\), \(p < .001\),
\(\eta^2_G = .070\). Importantly, the interaction of \emph{block number}
and \emph{FOC transition status} was significant,
\(F(5.9, 353.81) = 3.23\), \(\mathit{MSE} = 7.85\), \(p = .004\),
\(\eta^2_G = .012\), indicating that the difference in error rates
between regular and irregular transitions increased across blocks,
consistent with the learning effect obtained for reaction times.

\subsubsection{Generation task}\label{generation-task}

In a second step, we investigated how learned knowledge was expressed in
the generation task. We analyzed generation performance by fitting two
hierarchical models, \(\mathcal{M}_1\) and \(\mathcal{M}_2\).
\(\mathcal{M}_1\) allows the automatic and controlled processes to vary
between inclusion and exclusion, but it assumes that participants
acquired only implicit knowledge during the SRTT, and that revealing
explicit knowledge after the SRTT did not affect implicit knowledge.
\(\mathcal{M}_2\) is a hierarchical extension of the classical PD model
that enforces the invariance assumption. We computed model fit
statistics to test whether each model could account for the means,
\(T_{A1}\), and covariances, \(T_{B1}\), of the observed frequencies. We
compared both models using the DIC statistic that provides a combined
assessment of parsimony and goodness of fit and penalizes models for
unnecessary complexity. Parameter estimates from model \(\mathcal{M}_1\)
were used to address the invariance assumptions, directly. The first
trial of a block as well as any response repetitions were excluded from
all generation task analyses.

The model checks for model \(\mathcal{M}_1\) were satisfactory,
\[T_{A1}^{observed} = 491.06, T_{A1}^{expected} = 469.94, p = .369,\]~
\[T_{B1}^{observed} = 9.05, T_{B1}^{expected} = 6.95, p = .366.\] In
contrast, the model checks for model \(\mathcal{M}_2\) revealed
significant deviations of the model's predictions from the data,
\[T_{A1}^{observed} = 1,092.06, T_{A1}^{expected} = 473.88, p = .002,\]~
\[T_{B1}^{observed} = 190.05, T_{B1}^{expected} = 6.93, p < .001.\]

Model \(\mathcal{M}_1\) attained a DIC value of 25,293.45 and clearly
outperformed model \(\mathcal{M}_2\) that attained a DIC value of
25,891.74. This implies that the auxiliary assumptions we introduced to
\(\mathcal{M}_1\) were much less problematic than the invariance
assumption. Moreover, the standard PD model enforcing the invariance
assumption was not able to account for the data.



Figure 3 shows the parameter estimates obtained from model
\(\mathcal{M}_1\). Figure 4 shows that the invariance assumption for the
automatic process was violated with \(A_I > A_E\), 95\% CI {[}.01,
.03{]}, and a Bayesian \(p < .001\) (\(p = .360\) for revealed
transitions). The invariance assumption for the controlled process was
also violated with \(C_I > C_E\), 95\% CI {[}.08, .54{]}, and a Bayesian
\(p = .003\).

\subsection{Discussion}\label{discussion}

The experimental manipulations had the expected results: Based on the
SRTT results, we can conclude that participants acquired sequence
knowledge during learning. In addition, explicit knowledge about one of
the six transitions had a clear effect on generation performance for
that transition.

The extended process-dissociation model \(\mathcal{M}_1\) revealed a
violation of the invariance assumptions for both the controlled process
(i.e., \(C_I > C_E\)) and the automatic process (i.e., \(A_I > A_E\)).
Model \(\mathcal{M}_1\) rested on two auxiliary assumptions: It was
assumed that controlled processes were not affected by learning
material, and that automatic processes were not affected by the
manipulation of explicit knowledge (i.e., revealing a transition). Both
assumptions found support in the current data as they did not harm model
fit. Comparing model \(\mathcal{M}_1\) to a standard
process-dissociation model \(\mathcal{M}_2\) that did not impose these
assumptions but instead imposed the invariance assumption, model
\(\mathcal{M}_1\) was strongly favored by the DIC.

Despite the fact that the above auxiliary assumptions could be upheld in
model comparison, and that the incorporating model was well able to
account for the data, it may nevertheless still be the case that
violations have biased parameter estimates. To assess this possibility,
we used the questionnaire data to assess the assumption that the
controlled process was not affected by learning material (i.e., that
participants did not acquire explicit knowledge). Specifically, we
excluded any transitions that participants reported in their explicit
description of the sequence (while keeping the revealed transitions). If
participants had in fact acquired explicit knowledge about nonrevealed
transitions during learning, they may have used this knowledge to
generate more regular transitions under inclusion than exclusion.
Because of our assumption that \(C=0\) for nonrevealed transitions, this
performance difference would have been reflected in greater estimates of
implicit knowledge under inclusion than exclusion, and might account for
the observed \(A_{I}>A_{E}\) pattern. If the acquired explicit knowledge
was indeed the cause of the invariance violation, excluding the
transitions for which knowledge was reported should make the violation
disappear. To the contrary, excluding all correctly reported transitions
(9.04\% of cases) did not affect the pattern of results.\footnote{Of the
  reported (nonrevealed) transitions, only approximately 25.47\% were
  indeed regular transitions. After excluding \emph{all} reported
  transitions regardless of whether they reflect correct knowledge or
  not (27.55\% of cases), the invariance violation was descriptively
  unchanged but no longer statistically significant, Bayesian
  \(p = .221\).} This confirms the above conclusions that the auxiliary
assumptions can be upheld.

Taken together, these findings suggest that the invariance assumption
was violated for both the automatic and the controlled process.
Invariance of the automatic process was significantly violated for
nonrevealed but not for revealed transitions. This may be due to the
small magnitude of the violation effect and the relatively small number
of revealed (as compared to nonrevealed) transitions. The magnitude of
the invariance violation was much greater for the controlled process:
Explicit knowledge was used to a greater degree under inclusion than
exclusion instructions.

\section{Experiment 2}\label{experiment-2}

Experiment 1 showed that the invariance assumption was violated for both
automatic and controlled processes. The main goal of Experiment 2 was to
replicate the previous findings and extend them to second-order
conditional (SOC) material.

A secondary goal was to explore whether different amounts of implicit
knowledge are acquired with \emph{mixed} versus \emph{pure} SOC
material. Previous studies of the SRTT using a PD generation task have
employed a 12-item-sequence of four response locations (e.g.,
Destrebecqz \& Cleeremans, 2001; Wilkinson \& Shanks, 2004). Analyzing
these sequences more closely, it is evident that they did not only
contain second order information (i.e., the last two locations predict
the next location), but they also incorporate lower-order information:
Direct repetitions never occur, reversals occur below chance (i.e.,
1/12, whereas chance level would equal \(1/3\) given that repetitions
are prohibited), and the last location of a triplet \(L_3\) is not
independent of the first location \(L_1\) (e.g., for SOC1,
\(p(L_3 = 2 | L_1 = 3) = 2/3\)). In other words, in two out of three
cases, the third location of a triplet can be predicted by the first
location of a triplet alone. It is plausible that participants are able
to learn this lower-order information, and that learning effects may not
(only) be based on second-order information (cf., Reed \& Johnson,
1994). To investigate this possibility, Experiment 2 implemented two
types of probabilistic material: A \emph{mixed SOC} material that
incorporated both second-order and lower-order types of information, and
another \emph{pure SOC} material that only followed a second-order
regularity.

\subsection{Method}\label{method-1}

\subsubsection{Design}\label{design-1}

The study realized a 3 (\emph{material}: random, mixed SOC, pure SOC)
\(\times\) 2 (\emph{explicit knowledge}: no transition revealed vs.~two
transitions revealed) \(\times\) 2 (\emph{PD instruction}: inclusion
vs.~exclusion) \(\times\) 2 (\emph{block order}: inclusion first
vs.~exclusion first) design with repeated measures on the \emph{PD
instruction} factor.

\subsubsection{Participants}\label{participants-1}

One hundred and seventy-nine participants (120 women) aged between 18
and 58 years (\(M = 22.8\) years) completed the study. Most were
undergraduates from Heinrich-Heine-Universität Düsseldorf. Data from 8
participants were excluded from generation task analyses because they
had received erroneous exclusion instructions. Participants were
randomly assigned to experimental conditions. They received either
course credit or 3.50 Euro for their participation.

\subsubsection{Materials}\label{materials-1}

We implemented three different types of material:

\begin{itemize}
\tightlist
\item
  A \emph{random} sequence was randomly generated for each participant
  anew by drawing with replacement from a uniform distribution of six
  response locations.
\item
  A \emph{mixed SOC} sequence incorporated two types of information:
  First, the third location of a triplet was conditional upon the first
  two locations. Second, within such regular triplets, given a fixed
  first-position location, there was one highly probable third-position
  location and two somewhat less probable third-position locations; the
  other three response locations never occurred for this first-position
  location.
\item
  A \emph{pure SOC} sequence followed only the second-order regularity.
\end{itemize}

In both probabilistic materials (\emph{mixed} and \emph{pure} SOC),
87.5\% of trials adhered to the second-order regularity, which was
individually and randomly selected for each participant anew. In all
conditions, the material adhered to the following (additional)
restrictions: (1) there were no direct repetitions of response
locations, and (2) there were no response location reversals (i.e.,
1-2-1). To compute the dependent variable in the generation task (i.e.,
the number of rule-adhering triplets), for both \emph{probabilistic}
groups, we used the second-order sequence that was used to generate each
participant's materials. For the \emph{random} group, there is no
\enquote{correct} sequence and we again computed an individual criterion
sequence for each participant. For convenience, we did not generate all
possible second-order sequences for these participants (as we did for
first-order materials in Experiment 1), but chose to use individual
criterion sequences that were randomly generated similar to the
\emph{pure SOC} material.

\subsubsection{Procedure}\label{procedure-1}

The experimental procedure closely followed that of Experiment 1: In the
acquisition task, participants performed a SRTT consisting of 8 blocks
with 180 trials each (for a total of 1,440 responses). The
response-stimulus interval (RSI) was \(0~\text{ms}\). Following the SRTT
phase, participants were told that stimulus locations during the SRTT
followed some underlying sequential structure. They were then asked to
try to generate a short sequence of thirty locations that followed this
structure.

The generation task followed, with inclusion vs.~exclusion block order
counterbalanced. Deviating from Experiment 1, we fixed the number of
practice blocks that preceded both inclusion and exclusion task: Prior
to the inclusion task, three practice blocks involved inclusion
instructions; prior to the exclusion task, the first and second practice
block involved inclusion instructions, and the third involved exclusion
instructions. Before working on practice blocks, two transitions were
revealed to one half of the participants.

Upon completing the computerized task, participants were asked to
complete a questionnaire containing the following items: (1)
\enquote{Did you notice anything special working on the task? Please
mention anything that comes to your mind.}, (2) \enquote{One of the
tasks mentioned a sequence in which the squares lit up during the first
part of the study. In one of the experimental conditions, the squares
did indeed follow a specific sequence. Do you think you were in this
condition or not?}, (3) \enquote{How confident are you (in \%)?}, (4)
\enquote{Can you describe the sequence in detail?}. Subsequently,
participants were asked to indicate, for ten first-order transitions,
the next three keys in the sequence on a printed keyboard layout. The
first-order transitions were individually selected for each participant
so that each participant had the chance to express full explicit
knowledge about the second-order regularity.

\subsection{Data analysis}\label{data-analysis-1}

For the model-based analyses, models \(\mathcal{M}_1\) and
\(\mathcal{M}_2\) were analogous to those used in Experiment 1 (see
Appendix for detail).

\subsection{Results}\label{results-1}

We first analyzed reaction times and error rates during the SRT task to
determine whether sequence knowledge had been acquired during the task.
Next, we analyzed generation task performance using hierarchical PD
models.

\subsubsection{Acquisition task}\label{acquisition-task-1}

If participants acquired sequence knowledge from probabilistic
materials, we expect a performance advantage for regular over irregular
transitions, reflected in reduced RT and/or error rate. If this
advantage is due to learning, it is expected to increase over SRTT
blocks. If participants are able to learn lower-order information that
is only present in \emph{mixed SOC} material, the advantage is expected
to be greater in \emph{mixed SOC} material compared to \emph{pure SOC}.
If participants are able to learn second-order information, a
performance advantage is to be expected not only in \emph{mixed SOC} but
also in \emph{pure SOC} material.

\paragraph{Reaction times}\label{reaction-times-1}



For all RT analyses, we excluded the first two trials of each block
because the first two locations cannot be predicted, as well as error
trials, trials succeeding an error, reactions faster than 50 ms and
slower than 1,000 ms. Figure 5 shows reaction times during acquisition.

We conducted a 3 (\emph{Material}: random vs.~pure SOC vs.~mixed SOC)
\(\times\) 2 (\emph{Transition status}: regular vs.~irregular SOC)
\(\times\) 8 (\emph{Block number}) ANOVA with repeated measures on the
last two factors that revealed a main effect of \emph{block number},
\(F(4.46, 780.51) = 41.53\), \(\mathit{MSE} = 1,515.93\), \(p < .001\),
\(\eta^2_G = .020\), reflecting decreasing RT over blocks; a main effect
of \emph{transition status}, \(F(1, 175) = 40.02\),
\(\mathit{MSE} = 582.10\), \(p < .001\), \(\eta^2_G = .002\), reflecting
an RT advantage for regular transitions; and an interaction of
\emph{block number} and \emph{transition status},
\(F(6.39, 1118.42) = 2.81\), \(\mathit{MSE} = 439.60\), \(p = .009\),
\(\eta^2_G = .001\), reflecting the finding that the RT advantage for
regular transitions increased over block (i.e., the sequence learning
effect). We also found an interaction of \emph{material} and
\emph{transition status}, \(F(2, 175) = 7.40\),
\(\mathit{MSE} = 582.10\), \(p = .001\), \(\eta^2_G = .001\), reflecting
the finding that the effect of \emph{transition status} was absent in
the random material group, \(F(1, 58) = 0.44\),
\(\mathit{MSE} = 380.19\), \(p = .510\), \(\eta^2_G = .000\); trivially,
no sequence knowledge was learned from random material.

The three-way interaction was not significant,
\(F(12.78, 1118.42) = 0.92\), \(\mathit{MSE} = 439.60\), \(p = .535\),
\(\eta^2_G = .000\), suggesting that the sequence-learning effect did
not differ across material groups. We conducted separate analyses to
probe for sequence-learning effects in each material condition.
Analyzing only the random material group revealed only a main effect of
\emph{block number}, \(F(3.82, 221.55) = 15.74\),
\(\mathit{MSE} = 1,484.04\), \(p < .001\), \(\eta^2_G = .020\) (all
other \emph{p}s \textgreater{} .05). In the \emph{pure SOC} group, in
contrast, a main effect of \emph{block number},
\(F(3.96, 229.51) = 12.04\), \(\mathit{MSE} = 2,038.65\), \(p < .001\),
\(\eta^2_G = .019\), was accompanied by a main effect of
\emph{transition status}, \(F(1, 58) = 28.48\),
\(\mathit{MSE} = 637.73\), \(p < .001\), \(\eta^2_G = .004\), and an
interaction of both factors, \(F(6.03, 349.61) = 2.47\),
\(\mathit{MSE} = 530.13\), \(p = .023\), \(\eta^2_G = .002\), reflecting
a sequence learning effect on RT.

In the \emph{mixed SOC} group, we obtained only main effects of
\emph{block number}, \(F(4.91, 289.7) = 15.95\),
\(\mathit{MSE} = 1,334.22\), \(p < .001\), \(\eta^2_G = .024\), and of
\emph{transition status}, \(F(1, 59) = 18.83\),
\(\mathit{MSE} = 725.90\), \(p < .001\), \(\eta^2_G = .003\), but the
interaction of \emph{block number} and \emph{transition status} was not
significant, \(F(5.74, 338.77) = 1.15\), \(\mathit{MSE} = 571.40\),
\(p = .331\), \(\eta^2_G = .001\). This is despite the fact that the
effect of transition status is also likely to be a result of sequence
learning, and it is of similar magnitude to that obtained in the pure
SOC group. The notion that both learning effects are similar was also
supported by a joint analysis of the pure SOC and mixed SOC groups: The
two-way interaction between block number and transition status was
significant, \(F(6.39, 1118.42) = 2.81\), \(\mathit{MSE} = 439.60\),
\(p = .009\), \(\eta^2_G = .001\), but the three-way-interaction of
\emph{material}, \emph{block number}, and \emph{transition status} was
not significant, \(F(12.78, 1118.42) = 0.92\),
\(\mathit{MSE} = 439.60\), \(p = .535\), \(\eta^2_G = .000\). Taken
together, we interpret these findings to show that the learning effect
in the mixed SOC group was comparable to that observed in the pure SOC
group but too small to reach significance in a separate analysis.

\paragraph{Error rates}\label{error-rates-1}



For all analyses of error rates, we excluded the first two trials of
each block. Figure 6 shows error rates during acquisition. We conducted
a 3 (\emph{Material}: Random vs.~mixed SOC vs.~pure SOC) \(\times\) 8
(\emph{Block number}) \(\times\) 2 (\emph{SOC transition status}:
regular vs.~irregular) ANOVA with repeated measures on the last two
factors that revealed a main effect of \emph{block number},
\(F(3.66, 644.87) = 3.78\), \(\mathit{MSE} = 39.10\), \(p = .006\),
\(\eta^2_G = .008\), reflecting increasing error rates over blocks, and
a main effect of \emph{transition status}, \(F(1, 176) = 16.14\),
\(\mathit{MSE} = 9.08\), \(p < .001\), \(\eta^2_G = .002\), reflecting
an accuracy advantage for regular transitions. The interaction of
\emph{material} and \emph{transition status} was not significant,
\(F(2, 176) = 2.66\), \(\mathit{MSE} = 9.08\), \(p = .073\),
\(\eta^2_G = .001\),

Separate analyses yielded no significant effects in the random material
group (all \emph{p}s \textgreater{} .05). Importantly, an effect of
\emph{transition status} was clearly absent from the random material
group, \(F(1, 58) = 0.62\), \(\mathit{MSE} = 7.68\), \(p = .433\),
\(\eta^2_G = .000\). In the \emph{mixed SOC} group, a main effect of
\emph{block number} was found, \(F(5.66, 334.01) = 2.96\),
\(\mathit{MSE} = 15.46\), \(p = .009\), \(\eta^2_G = .017\), along with
a main effect of \emph{transition status}, \(F(1, 59) = 12.88\),
\(\mathit{MSE} = 11.29\), \(p = .001\), \(\eta^2_G = .009\), reflecting
higher error rates for irregular than for regular transitions. Finally,
in the \emph{pure SOC} group, block number did not affect error rates,
\(F(1.87, 110.6) = 1.72\), \(\mathit{MSE} = 133.60\), \(p = .185\),
\(\eta^2_G = .011\); but a main effect of \emph{transition status} was
also found, \(F(1, 59) = 5.55\), \(\mathit{MSE} = 8.24\), \(p = .022\),
\(\eta^2_G = .001\), reflecting higher error rates for irregular than
regular transitions.

Taken together, error rates mirror RTs in that they also reflect a
performance advantage for regular transitions in the mixed and pure SOC
groups that was not evident in the random control group. Deviating from
the RT result pattern, this advantage did not reliably increase across
blocks.

\subsubsection{Generation task}\label{generation-task-1}

We analyzed generation performance by fitting the two hierarchical
models \(\mathcal{M}_1\) and \(\mathcal{M}_2\) that we introduced above
to the data from Experiment 2. For both models, we computed model fit
statistics to assess whether each model could account for the data; we
then compared both models using the DIC. Parameter estimates from model
\(\mathcal{M}_1\) were then used to address the invariance assumptions
directly. The first two trials of a block as well as any response
repetitions and reversals were excluded from all generation task
analyses.

The model checks for model \(\mathcal{M}_1\) were satisfactory,
\[T_{A1}^{observed} = 692.77, T_{A1}^{expected} = 653.45, p = .291,\]~
\[T_{B1}^{observed} = 8.44, T_{B1}^{expected} = 6.04, p = .292.\] In
contrast, the model checks for model \(\mathcal{M}_2\) revealed
significant deviations of the model's predictions from the data,
\[T_{A1}^{observed} = 1,077.52, T_{A1}^{expected} = 652.79, p = .003,\]~
\[T_{B1}^{observed} = 49.97, T_{B1}^{expected} = 6.06, p < .001.\]

Model \(\mathcal{M}_1\) attained a DIC value of 38,907.43 and
outperformed model \(\mathcal{M}_2\) that attained a DIC value of
39,210.66. This implies that our auxiliary assumptions that we
introduced to make model \(\mathcal{M}_1\) identifiable (i.e., that
participants did not acquire explicit knowledge during training, and
that revealing explicit knowledge about a transition did not affect
implicit knowledge) were less problematic than the invariance
assumption. Moreover, the standard PD model enforcing the invariance
assumption was not able to account for the data.



Figure 7 shows the parameter estimates obtained from model
\(\mathcal{M}_1\). Figure 8 shows that the invariance assumption for
controlled processes was again violated with \(C_I > C_E\), 95\% CI
{[}.27, .63{]}, Bayesian \(p < .001\). The invariance assumption for
automatic processes could be upheld, 95\% CI {[}-.01, .01{]}, Bayesian
\(p = .638\) for non-revealed transitions and 95\% CI {[}-.10, .05{]},
\(p = .763\) for revealed transitions.

\subsection{Discussion}\label{discussion-1}

The experimental manipulations had the expected results: Based on the
SRTT results, we can conclude that participants acquired some (albeit
weak) sequence knowledge during learning. In addition, generation
performance was clearly affected by instructed explicit knowledge.

An extended process-dissociation model \(\mathcal{M}_1\) revealed a
violation of the invariance assumption for controlled processes with
\(C_I > C_E\). The invariance assumption for automatic processes could
be upheld. Model \(\mathcal{M}_1\) rested on two auxiliary assumptions:
It was assumed that controlled processes were not affected by learning
material, and that automatic processes were not affected by the
manipulation of explicit knowledge (i.e., revealing a transition). Both
assumptions found support in the current data as they did not harm model
fit. Importantly, comparing model \(\mathcal{M}_1\) to a standard
process-dissociation model \(\mathcal{M}_2\) that did not impose these
assumptions but left the invariance assumption intact, model
\(\mathcal{M}_1\) was strongly favored by the DIC.

Regarding our secondary goal to explore whether different amounts of
sequence knowledge are acquired from mixed versus pure second-order
conditional material, we did not find evidence for a difference between
these two types of material in the SRTT. This may well be due to the
overall low levels of acquired sequence knowledge in the present study.
Clearly, the present data are not strong enough to rule out such
differences; this question requires further study.

\section{Experiment 3}\label{experiment-3}

Experiments 1 and 2 found a violation of the invariance assumption,
suggesting that the interpretation of the parametric PD approach may be
problematic. In particular, the results of Experiments 1 and 2
consistently suggest that the PD parameters may not yield an exhaustive
measure of explicit knowledge: The degree to which participants made use
of their explicit knowledge varied between the inclusion and exclusion
tasks. Experiment 3 aimed at obtaining a better understanding of this
invariance violation and its effect on the interpretation of generation
task performance in the PD framework.

Whereas the invariance violations clearly threaten the validity of the
PD model, they may yet turn out to be uncritical for an ordinal
interpretation of PD data that has often been used in applications
(i.e., a comparison of inclusion versus exclusion performance, and of
exclusion versus baseline performance). Even if the (independence and)
invariance assumptions do not hold, the general approach of drawing
conclusions about the underlying processes by comparing performance
between inclusion and exclusion conditions may not be entirely
invalidated: It has been formally shown that an ordinal interpretation
of PD findings does not rely on parametric assumptions (Hirshman, 2004).
However, the ordinal PD approach does assume that baseline performance
is identical in the inclusion and exclusion tasks -- an assumption that
has been shown to be violated at least in some cases (Stahl et al.,
2015). More critically for the present question of invariance, it also
assumes (not only that inclusion performance increases but also) that
exclusion performance monotonically decreases with increasing explicit
knowledge. Note that, if (contrary to this assumption) explicit
knowledge does not affect exclusion performance at all, the ordinal PD
approach may technically still be used. However, the results would be
misleading if a difference in explicit (but not implicit) knowledge
between two conditions led to a difference in inclusion but not in
exclusion performance. In this case, the ordinal PD would suggest that
the two conditions differ in explicit \emph{and implicit} knowledge
(Hirshman, 2004, Data Pattern I). In other words, for the ordinal PD
approach to yield valid results, exclusion performance must fall below
baseline when explicit knowledge is present (this would yield Hirshman's
Data Pattern IV, which indicates an increase in explicit knowledge).
Therefore, a critical empirical test for the ordinal PD approach is
whether, and under which conditions, participants are able to use
explicit knowledge to suppress generation below baseline levels under
exclusion conditions.

A critical precondition for the expression of explicit knowledge may be
the opportunity for practice with the generation task. Experiment 3
investigated the effect of practice on the expression of explicit
knowledge: In Experiments 1 and 2, participants were given the
opportunity to practice inclusion and exclusion of the instructed
explicit knowledge, but nevertheless failed to approach ceiling (floor)
performance in the inclusion (exclusion) conditions. In Experiment 3, we
investigated whether such transition-specific practice can help reduce
the invariance violation by comparing practiced with non-practiced
transitions. We also wanted to more directly investigate transfer of
practice to unpracticed transitions about which participants had
explicit knowledge. We therefore manipulated the number of revealed
transitions, and whether these revealed transitions were revealed prior
to or after practice blocks, both between and within subjects. We
realized five between-subjects conditions:

\begin{enumerate}
\def\labelenumi{\arabic{enumi}.}
\tightlist
\item
  In the \emph{Control} group, no explicit knowledge was revealed to
  participants.
\end{enumerate}

\begin{enumerate}
\def\labelenumi{\arabic{enumi}.}
\setcounter{enumi}{1}
\tightlist
\item
  In the \emph{No-Practice} group, one transition was revealed
  immediately before the first generation block, but \emph{after} the
  practice blocks that preceded the first generation block. To avoid
  carry-over of practice effects from the first generation block, a
  different non-practiced transition was revealed after the second
  practice blocks and immediately preceding the second generation block.
\end{enumerate}

\begin{enumerate}
\def\labelenumi{\arabic{enumi}.}
\setcounter{enumi}{2}
\tightlist
\item
  In the \emph{Unspecific-Practice} group, one transition was revealed
  to participants \emph{after} practive, immediately before each
  generation block (as in the \emph{No-Practice} group). In the third
  practice block before the exclusion task, participants were asked to
  inhibit a specific response location (i.e., they were asked \emph{not}
  to use the \(5^{th}\) location/\(N\) key).
\end{enumerate}

\begin{enumerate}
\def\labelenumi{\arabic{enumi}.}
\setcounter{enumi}{3}
\tightlist
\item
  In the \emph{Practice} group, one transition was revealed to
  participants immediately \emph{before} the practice blocks.
  Participants were encouraged to include (exclude) the revealed
  transition during practice and in the generation block.
\end{enumerate}

\begin{enumerate}
\def\labelenumi{\arabic{enumi}.}
\setcounter{enumi}{4}
\tightlist
\item
  In the \emph{Transfer} group, information about two transitions was
  revealed; one of them was non-practiced (as in the No-Practice group),
  the other one practiced (as in the Practice group). The practiced
  transition was revealed before the first practice blocks. After these
  practice blocks, the second (non-practiced) transition was revealed
  immediately before the first generation block. The practiced
  transition was again named before participants worked on the practice
  blocks of the second generation phase. After these practice blocks, a
  second non-practiced transition was revealed immediately before the
  second generation block.
\end{enumerate}

The Control and Practice groups were identical to Experiment 1; the
other groups extended the previous design. This allowed us to assess
generation performance for three main transition types; (1)
\emph{non-revealed} transitions, (2) transitions that were revealed but
remained \emph{non-practiced}, and (3) transitions that were revealed
and \emph{practiced} in the practice blocks.\footnote{In the second
  generation block of the No-Practice, Unspecific-Practice, and Transfer
  groups, a fourth transition type can be distinguished: Transitions
  that were revealed but non-practiced before the first generation
  block. Because participants included (or excluded) these transitions
  in the previous (i.e., the first) generation block, performance on
  these transitions should be more similar to practiced than to
  non-practiced transitions in the second block.}

A comparison of \emph{non-revealed} with (revealed but)
\emph{non-practiced} transitions allows us to assess the degree to which
participants can spontaneously make use of their explicit knowledge in
the generation task. Comparing \emph{non-practiced} with
\emph{practiced} transitions allowed us to assess whether specific
inclusion/exclusion practice could increase the use of explicit
knowledge to a comparable level in the inclusion and exclusion tasks
(i.e., eliminate the violation of invariance). We also compared whether
performance for revealed but \emph{non-practiced} transitions differs
between the No-Practice and Transfer groups, as would be expected if the
effect of specific practice transfers to non-practiced explicit
knowledge. Finally, we explored whether, in the Unspecific-Practice
group, unspecific inhibition practice affects performance for both
revealed but \emph{non-practiced} and/or \emph{non-revealed}
transitions.

In sum, we hypothesized that, as in Experiment 1, the invariance
assumption for the controlled process would be violated. Relatedly, we
aimed at replicating the finding that possessing explicit knowledge
would not be sufficient for its expression in the generation task.
Specifically, (a) explicit knowledge without practice
(\emph{No-Practice} group) should not lead to below-chance exclusion
performance, and (b) this should also hold for non-practiced transitions
for participants who practiced another transition (\emph{Transfer}
group). We had no clear hypothesis regarding the unspecific
response-inhibition practice, but wanted to explore whether it would be
as effective as transition-specific exclusion practice in improving the
validity of the generation task as a measure of explicit knowledge.
Finally, Experiment 3 addresses a possible artifactual effect of
practice on the invariance violation. Specifically, the practice blocks
administered in Experiments 1 and 2 might have \emph{caused} the
invariance violation. If this was the case, then invariance should not
be violated in the absence of such practice.

\subsection{Method}\label{method-2}

\subsubsection{Design}\label{design-2}

The study realized a 5 (\emph{Practice group}: Control, No-Practice,
Unspecific-Practice, Practice, Transfer) \(\times\) 2 (\emph{PD
instruction}: inclusion vs.~exclusion) \(\times\) 2 (\emph{block order}:
inclusion first vs.~exclusion first) design with repeated measures on
the \emph{PD instruction} factor.

\subsubsection{Participants}\label{participants-2}

One hundred and forty-seven participants (113 women) aged between 17 and
55 years (\(M = 23.7\) years) completed the study. Most were
undergraduates from Heinrich-Heine-Universität Düsseldorf. Participants
were randomly assigned to experimental conditions. They received either
course credit or 3.50 Euro for their participation.

\subsubsection{Materials and Procedure}\label{materials-and-procedure}

The experimental procedure closely followed Experiment 1. During an SRTT
consisting of 8 blocks with 144 trials each (for a total of 1,152
responses), participants in all conditions were trained with a
\emph{probabilistic} sequence similar to the one used in Experiment 1.
After the SRTT, participants were informed about the underlying
sequential structure of stimulus locations and asked to generate a short
sequence of six key presses that followed this (unspecified) structure.

The generation task followed, with counterbalanced order of inclusion
versus exclusion blocks. The number of practice blocks was held constant
(in contrast to Experiment 1, where it depended on performance). Upon
completing the computerized task, participants answered the same
questionnaire as in Experiment 1.

\subsubsection{Data analysis}\label{data-analysis-2}

Given that model \(\mathcal{M}_2\) failed to fit the data from both
Experiments 1 and 2 and model, we fitted only model \(\mathcal{M}_1\)
and used posterior analyses to evaluate the invariance assumption. For
the model-based analyses, we adapted the equations from Experiment 1 to
the design of Experiment 3 (which did not contain experimental groups
with random material).

In order to accommodate for the more complex design, we used a model
specification that allowed for participant and item (i.e., transition)
effects and their interactions by estimating fixed effects for each
transition type plus individual participants' deviations from these
effects. The model equations of model \(\mathcal{M}_1\) are given by:

\[
  C_{ijm} = \begin{cases}
    \Phi(\mu_{jlm}^{(C)} + \delta_{ijm}^{(C)}) & \text{if } j \epsilon 1, 2 \text{ (item has been revealed \& practiced, revealed \& non-practiced)}\\
                                              0 & \text{if }j=3 \text{ (item has not been revealed)}\\
    \end{cases}
\] and \[
  A_{imt} = \Phi(\mu_{mt}^{(A)} + \delta_{imt}^{(A)})
\]

where \(\mu_{jlm}^{(C)}\) is the fixed effect of transition type \(j\)
(non-revealed, revealed \& practiced, revealed \& non-practiced) in
condition \(l\) and \emph{PD instruction} condition \(m\) on controlled
processes, and \(\delta_{ijm}^{(C)}\) is the \(i\)th participant's
deviation from the corresponding mean. Accordingly, \(\mu_{mt}^{(A)}\)
is the fixed effect of \emph{PD instruction} condition \(m\) and
transition \(t\) on automatic processes, and \(\delta_{imt}^{(A)}\) is
the \(i\)th participant's deviation from the corresponding mean.

Model \(\mathcal{M}_1\) imposes two auxiliary assumptions: First, it
assumed that no explicit knowledge has been acquired during the SRT
phase (i.e., \(C=0\) for non-revealed transitions). Second, it assumed
that revealing sequence knowledge did not affect automatic processes
(i.e., \(A\) does not vary as a function of the between-subjects
manipulation of explicit knowledge, index \(l\)). Both auxiliary
assumptions were tested by posterior predictive checks. In addition to
reporting \(T_{A1}\) and \(T_{B1}\) as in the previous studies, we
calculated additional model check statistics \(T_{A2}\), which
summarizes how well the model describes the item-wise category counts
(aggregated over participants), and \(T_{A3}\), which summarizes how
well the model describes the category counts per participant-item
combination; finally, the additional statistic \(T_{B2}\) summarizes how
well the model describes the variances and covariances introduced by
items. We also calculated the posterior differences \(C_I - C_E\) and
\(A_I - A_E\) to more directly test the invariance assumption.

\subsection{Results}\label{results-2}

We first analyzed the performance data from the SRT task in a
traditional way to determine whether sequence knowledge had been
acquired during the task. Next, we analyzed generation task performance
using a hierarchical PD model. Finally, to test our predictions
regarding the different effects of practice in a model-free manner we
analyzed generation performance for transitions that were revealed.

\subsubsection{Acquisition task}\label{acquisition-task-2}

If participants acquired knowledge about the regularity underlying the
sequence of key presses, we expect a performance advantage for regular
over irregular transitions, reflected in reduced RT and/or error rate.
If this advantage is due to learning, it is expected to increase over
SRTT blocks.

\paragraph{Reaction times}\label{reaction-times-2}



For all RT analyses, we excluded the first trial of each block as well
as trials with errors, trials succeeding an error, reactions faster than
50 ms and those slower than 1,000 ms. Figure 9 shows reaction times
during acquisition.

We conducted a 8 (\emph{Block number}) \(\times\) 2 (\emph{FOC
transition status}: regular vs.~irregular) repeated-measures ANOVA.
There was a main effect of \emph{block number},
\(F(4.26, 622.43) = 96.37\), \(\mathit{MSE} = 1,034.51\), \(p < .001\),
\(\eta^2_G = .059\), with RT decreasing over blocks. There also was a
main effect of \emph{FOC transitions status}, \(F(1, 146) = 573.39\),
\(\mathit{MSE} = 841.10\), \(p < .001\), \(\eta^2_G = .066\), reflecting
faster responses to regular than to irregular transitions. The
interaction of \emph{block} and \emph{FOC transition status} was also
significant, \(F(6.47, 945.32) = 58.60\), \(\mathit{MSE} = 176.46\),
\(p < .001\), \(\eta^2_G = .010\), reflecting the finding that the RT
advantage for regular transitions increased over blocks, which indicated
successful sequence learning.

\paragraph{Error rates}\label{error-rates-2}



For all analyses of error rates, we excluded the first trial of each
block. Figure 10 shows error rates during acquisition.

The pattern of findings was similar to that obtained for RT. We
conducted an 8 (\emph{Block number}) \(\times\) 2 (\emph{FOC transition
status}: regular vs.~irregular) repeated-measures ANOVA that revealed a
main effect of \emph{block number}, \(F(6.29, 917.65) = 8.35\),
\(\mathit{MSE} = 9.42\), \(p < .001\), \(\eta^2_G = .015\), reflecting
increasing error rates over blocks; and a main effect of \emph{FOC
transition status}, \(F(1, 146) = 188.88\), \(\mathit{MSE} = 11.92\),
\(p < .001\), \(\eta^2_G = .066\), reflecting an accuracy advantage
(i.e., lower error rates) for regular transitions. The interaction of
\emph{block number} and \emph{FOC transition status} was also
significant, \(F(6.53, 953.88) = 7.36\), \(\mathit{MSE} = 7.09\),
\(p < .001\), \(\eta^2_G = .011\), reflecting an increase of the
accuracy advantage for regular (as compared to irregular) transitions
over blocks, indicating successful sequence learning.

\subsubsection{Generation task}\label{generation-task-2}

We analyzed generation performance by fitting \(\mathcal{M}_1\) and
computed model fit statistics to assess whether the model can account
for the data. Parameter estimates from model \(\mathcal{M}_1\) were used
to address the invariance assumptions, directly. The first trial of a
block as well as any response repetitions were excluded from all
generation task analyses.

The model checks for model \(\mathcal{M}_1\) were satisfactory,
\[T_{A1}^{observed} = 35.97, T_{A1}^{expected} = 33.96, p = .322,\]~
\[T_{A2}^{observed} = 0.05, T_{A2}^{expected} = 0.05, p = .480,\]~
\[T_{A3}^{observed} = 1,763.79, T_{A3}^{expected} = 1,720.63, p = .372,\]~
\[T_{B1}^{observed} = 5.31, T_{B1}^{expected} = 4.62, p = .457,\]~
\[T_{B2}^{observed} = 3,852.65, T_{B2}^{expected} = 3,393.90, p = .464.\]

Figure 11 shows the parameter estimates obtained from model
\(\mathcal{M}_1\); while estimates of the automatic process were only
slightly above chance in both \emph{PD instruction} conditions,
estimates of the controlled process differ strongly between \emph{PD
instruction} conditions.

Figure 12 shows that the invariance assumption for automatic processes
was violated with \(A_I > A_E\), 95\% CI {[}.00, .03{]}, and Bayesian
\(p = .008\). For revealed and practiced transitions, the invariance
assumption was violated with \(C_I > C_E\), 95\% CI {[}.19, .63{]} and a
Bayesian \(p = .001\). For revealed but non-practiced transitions, the
invariance assumption was violated with \(C_I > C_E\), 95\% CI {[}.03,
.31{]} and a Bayesian \(p = .005\).



\subsubsection{Effects of practice on generation of revealed
transitions}\label{effects-of-practice-on-generation-of-revealed-transitions}

To test our predictions regarding the different effects of practice in a
model-free manner, we analyzed raw generation frequencies for only those
transitions about which explicit knowledge was revealed.



Figure 13 shows generation performance for revealed transitions. A 5
(\emph{Condition}: Control vs.~No-Practice vs.~Unspecific-Practice
vs.~Practice vs.~Transfer) \(\times\) 2 (\emph{Order}: Inclusion first
vs.~Exclusion first) \(\times\) 2 (\emph{PD instruction}: Inclusion
vs.~Exclusion) ANOVA revealed a nonsignificant main effect of
\emph{Condition}, \(F(3, 110) = 2.00\), \(\mathit{MSE} = 660.29\),
\(p = .119\), \(\eta^2_G = .028\), but a significant main effect of
\emph{PD instruction}, \(F(1, 110) = 243.88\),
\(\mathit{MSE} = 575.67\), \(p < .001\), \(\eta^2_G = .508\), and their
significant interaction, \(F(3, 110) = 5.59\),
\(\mathit{MSE} = 575.67\), \(p = .001\), \(\eta^2_G = .066\). The main
effect of \emph{PD instruction} reflects the clear influence of the
instructed explicit knowledge depicted in Figure 13. It is present in
all practice conditions but modulated by amount of knowledge and type of
practice (i.e., greater effects given specific practice): The effect was
greatest in the \emph{Transfer} group, \(t(29) = 14.84\), \(p < .001\),
\(d = 2.71\); somewhat smaller in the \emph{Practice} group,
\(t(28) = 9.79\), \(p < .001\), \(d = 1.82\); it was still smaller and
comparable without practice, \emph{No-practice} group, \(t(28) = 5.25\),
\(p < .001\), \(d = 0.97\); or with only unspecific practice,
\emph{Unspecific-practice} group, \(t(29) = 5.13\), \(p < .001\),
\(d = 0.94\).

We investigated this issue more closely in two sets of follow-up
analyses. Whereas the above findings support the hypothesis that
practice improves the degree to which explicit knowledge is expressed in
the generation task, it does not elucidate the mechanism by which this
occurs. One mechanism by which practice may improve performance is by
boosting the proportion of regular transitions in inclusion blocks.

\paragraph{Inclusion}\label{inclusion}



Inclusion performance for revealed transitions in the \emph{No-Practice}
and \emph{Practice} groups was analyzed as a function of practice
(practiced vs.~non-practiced), as depicted in Figure 14. Results showed
no effect of practice on generation performance, \(F(1, 56) = 0.21\),
\(\mathit{MSE} = 696.48\), \(p = .652\), \(\eta^2_G = .004\). Similarly,
when we compared inclusion performance for practiced vs.~non-practiced
transitions in the \emph{Transfer} group, there was no effect of
practice, \(F(1, 29) = 1.19\), \(\mathit{MSE} = 365.77\), \(p = .285\),
\(\eta^2_G = .014\). We conclude that practice did not affect inclusion
performance for revealed transitions.



\paragraph{Exclusion}\label{exclusion}

Next, we analyzed whether practice improves suppressing the regular
transition in the exclusion task. We hypothesized that, without
training, participants might not be able to suppress their generation of
regular transitions below the chance level in the exclusion task. To
test this hypothesis, we compared generation performance for the
revealed transitions between the \emph{No-Practice} and \emph{Practice}
groups, as depicted in the left panel of Figure 15. The expected
below-chance performance was not found in the data from both blocks:
Whereas the direction of effects was as expected, there was no deviation
from chance, neither for the practice condition, \(t(28) = -0.79\),
\(p = .219\), \(d = -0.15\), nor for the no-practice condition,
\(t(28) = 1.60\), \(p = .940\), \(d = 0.30\). However, the expected
pattern was found when only the first block was analyzed: Below-chance
performance was found for the practice condition, \(t(14) = -4.89\),
\(p < .001\), \(d = -1.26\), but not for the no-practice condition,
\(t(13) = 0.18\), \(p = .569\), \(d = 0.05\).

To more directly establish a practice effect, we next turned to the data
from the \emph{Transfer} group for a within-subjects comparison of
practiced and non-practiced transitions. In addition, we addressed the
transfer hypothesis: If specific training is required for each single
transition, the finding of at-chance exclusion performance should
replicate for non-practiced transitions in participants who practiced
another transition. In contrast, if training on one transition transfers
to other transitions, we should find below-chance performance for
non-practiced transitions in a parallel within-participants comparison
in the \emph{Transfer} group. As can be seen from the right panel of
Figure 15, generation performance was below chance for practiced,
\(t(29) = -9.60\), \(p < .001\), \(d = -1.75\), as well as for
non-practiced transitions, \(t(29) = -2.04\), \(p = .025\),
\(d = -0.37\), indicating transfer of exclusion practice from practiced
to non-practiced transitions.\footnote{Analyzing only the first block
  revealed the same pattern of results: Generation performance was below
  chance for practiced, \(t(14) = -5.42\), \(p < .001\), \(d = -1.40\),
  as well as for non-practiced transitions, \(t(14) = -4.56\),
  \(p < .001\), \(d = -1.18\).}

\subsection{Discussion}\label{discussion-2}

The experimental manipulations had the expected effects on implicit and
explicit sequence knowledge: Participants in Experiment 3 acquired
knowledge about the sequence, as expressed in RT- and accuracy
advantages for regular transition that increased over SRTT blocks.
Participants received different amounts of instructed explicit
knowledge, and they were able to express this knowledge in the
generation task, as revealed by the effect of \emph{PD instruction} on
generation of revealed transitions. However, performance differed across
groups (i.e., practice conditions), suggesting that specific exclusion
practice was beneficial to implementing PD instructions. Finally, even
with practice, inclusion performance did not reach ceiling and exclusion
performance did not reach floor levels, indicating that participants
were not able to exhaustively express their explicit knowledge in the
generation task.

The invariance assumption was again found to be violated for both
controlled and automatic processes. Explicit knowledge was expressed to
a greater degree in the inclusion than in the exclusion blocks of the
generation task. Generation practice improved the degree to which
explicit knowledge was expressed under exclusion instructions, but
invariance was violated despite repeated opportunities for practicing to
include/exclude a specific transition.

Results showed that practice increased the magnitude of the invariance
violation (i.e., the I-E difference). Importantly, however, this does
not imply that our evidence for invariance violation reflect an artifact
of practice. First, increasing the overall expression of explicit
knowledge (as suggested by parameter estimates, as well as the
below-chance exclusion performance under practice conditions) is of
course precisely the intended effect of practice. Second, only with
practice were participants sometimes able to suppress exclusion
performance below chance baselines (as required by the PD model). Third,
invariance was also violated in the absence of practice. We conclude
that invariance was violated because, overall, participants were not
able to refrain from using their explicit knowledge under exclusion
conditions (even if practiced). Perhaps more precisely, participants
failed to generate a sufficiently high proportion of irregular
transitions under exclusion conditions.

\section{General Discussion}\label{general-discussion}

\subsection{Summary of main findings}\label{summary-of-main-findings}

Process-dissociation assumes that the controlled and automatic process
are invariant under inclusion and exclusion instructions. In three
sequence-learning experiments, we tested whether this invariance
assumption holds in the generation task. The results show a consistent
pattern.

\subsubsection{Invariance of the controlled
process}\label{invariance-of-the-controlled-process}

The invariance assumption for explicit knowledge was consistently
violated, in first-order as well as second-order material, and despite
extensive opportunity for (as well as without) practice. In all cases,
explicit knowledge was expressed to a greater degree under inclusion
than under exclusion instructions: Participants succeeded in generating
the revealed transition under inclusion conditions, but failed to
refrain from generating that transition under exclusion conditions.
Specifically, under exclusion conditions, participants generated the
revealed transition at chance levels, instead of suppressing its
generation altogether as instructed.

Participants were largely unable to use their explicit knowledge to
suppress the proportion of regular transitions generated in the
exclusion task to levels below chance. Such below-chance generation
levels for revealed transitions were robustly found only for material
with a first-order regularity, and only in participants who had explicit
knowledge about (at least) two transitions and engaged in
generation-task practice specific to a given to-be-excluded transition
(Exp. 3, Transfer condition). In these participants, there was even some
evidence that below-chance exclusion performance transferred to
non-practiced explicit knowledge. However, transition-specific practice
was (necessary but) not sufficient for successful exclusion: Whereas
participants without such practice (i.e., the No-Practice and
Unspecific-Practice conditions of Exp.3) failed to reach below-chance
levels, participants with practice also failed to attain below-chance
levels under exclusion instructions if they worked on the inclusion task
first (i.e., Exp.3, Practice condition). Moreover, despite having
explicit knowledge as well as transition-specific generation-task
practice, participants were not able to exclude their explicit knowledge
to below-chance levels with a second-order conditional sequence (Exp.
2). Taken together, across three Experiments we obtained strong evidence
for a violation of invariance of the controlled process, and the results
of Experiment 3 suggest that this is due to a failure to suppress the
generation of regular transitions below chance levels.

\subsubsection{Invariance of the automatic
process}\label{invariance-of-the-automatic-process}

In Experiments 1 and 3 that used first-order conditional material, we
found evidence suggesting a violation of the invariance assumption for
implicit knowledge (no such evidence was found for the second-order
conditional material used in Experiment 2). If interpreted in a standard
PD framework, the inclusion-exclusion performance difference resulting
from this violation may lead to erroneous conclusions about the presence
of explicit knowledge (if such knowledge is indeed absent), or to
overestimation of the contribution of explicit knowledge. We believe
these findings of an inclusion-exclusion difference in estimates of the
automatic parameter should be interpreted with some caution, for at
least three reasons (see also the section Limitations below). First, the
finding was inconsistent and there are multiple possible causes of this
inconsistency: The lack of a violation in Experiment 2 may be due to
specific properties of the material, or it may be due to the fact that
sequence knowledge levels in that study were too low for differences in
its expression to be measurable.

Second, although robust and replicated, the violation was relatively
small (i.e., the \(A_{I}-A_{E}\) difference ranged between .01 and .03
in Exp.1, and between .00 and .03 in Exp.3). In the absence of
controlled influences, this would be equivalent to a difference between
inclusion and exclusion performance of approximately 2 percentage
points---an effect barely noticeable under typical conditions.

Third, it is unclear whether the observed invariance violation of
parameter \(A\) reflects implicit knowledge at all. Note that the
parameter for the automatic process captures the sum of all
non-controlled influences on generation performance. In particular, it
might reflect guessing strategies, and these may differ under inclusion
versus exclusion conditions (Stahl et al., 2015). In other words, the
above effect may reflect a violation of invariance of guessing or
response strategies instead of a violation of invariance of the
automatic expression of implicit knowledge. Taken together, we interpret
the finding as too weak to conclude that the invariance assumption is
violated also for the automatic process.

\subsection{Limitations and open
questions}\label{limitations-and-open-questions}

Before turning to the implications of the present findings, we discuss
potential limitations and address open questions.

\subsubsection{The invariance violation of the automatic process may
reflect learned explicit
knowledge}\label{the-invariance-violation-of-the-automatic-process-may-reflect-learned-explicit-knowledge}

Instead of being due to guessing, the inclusion-exclusion difference in
estimates of the automatic parameter may be due to explicit knowledge
acquired during learning. Such an effect, if present at all, is likely
to be small given that (1) the material was probabilistic and therefore
difficult to learn explicitly; (2) the model incorporating the
assumption that no learned explicit knowledge was learned fitted the
data well; and (3) the results were unchanged when we excluded the data
from transitions that participants (correctly) reproduced during
debriefing. However, we cannot exclude the possibility that small
amounts of explicit knowledge, obtained during the SRTT phase, may have
distorted our model's parameter estimates. This interpretation could
also account for the lack of such an effect in Experiment 2 given that
explicit knowledge was less likely to be learned from the more complex
second-order conditional material used in that study. If this were true,
then any differences between inclusion and exclusion that were
attributed by the model to an invariance violation of the implicit
process may in fact have been a consequence of residual explicit
knowledge that was not reflected in our debriefing questionnaire
(perhaps due to participants' conservative reporting criteria). For
validating the PD approach, we know of no other ways to address this
potential confound other than by controlling for explicit sequence
knowledge as assessed by independent measures; successful control is
then naturally limited by the validity of these independent measures.
This limitation is another reason for caution in interpreting the above
finding as evidence for a violation of invariance of the automatic
process. Note that it does not limit the interpretation of our main
finding of the invariance violation of the controlled process.

\subsubsection{The evidence for sequence learning was weak for SOC
material in Experiment
2}\label{the-evidence-for-sequence-learning-was-weak-for-soc-material-in-experiment-2}

As expected, second-order conditional material (Exp. 2) was more
difficult to learn than first-order conditional material (Expts. 1 \&
3). This was reflected here in the finding that (despite a 20\% increase
in learning trials) there was only weak evidence for sequence learning
in Experiment 2. Specifically, responses to regular transitions were
clearly faster and more accurate for both variants of the SOC materials,
but the interaction between regularity and training block, which is
critical for interpreting a performance advantage for regular
transitions as an effect of learning, was not significant. Clearly, an
even larger amount of SRTT training should be realized in future studies
using SOC materials. Yet, it is unlikely that the advantage for regular
transitions has any other causes than learning, given that it was absent
from the random condition, and that the effect could not be attributed
to properties of specific transitions because regularity of a transition
was randomized for each participant anew. Nevertheless, because evidence
for (implicit) sequence learning was not beyond doubt, it is not
warranted to interpret the modeling results as stringent tests of the
invariance assumption for the automatic process.

\subsubsection{Explicit knowledge learned via instruction may be
qualitatively different from acquired explicit
knowledge}\label{explicit-knowledge-learned-via-instruction-may-be-qualitatively-different-from-acquired-explicit-knowledge}

The present study manipulated explicit knowledge via instruction.
Although it is a common method (e.g., Liefooghe, Wenke, \& De Houwer,
2012) that has yielded important insights in other domains, one might
argue that explicit knowledge acquired via instruction is somehow
qualitatively different from explicit knowledge acquired during SRTT
training, and that therefore the present results do not speak to the
question of interest regarding the invariance of the expression of
acquired knowledge. We believe our manipulation to be valid for the
following reasons. First, the instructed explicit knowledge communicated
the same proposition about the sequence that participants would have
acquired during SRTT training (i.e., that a specific location was
regularly followed by another location). Second, we took precautions to
avoid any inconsistency or conflict with learned sequence knowledge:
Transitions that were revealed to participants were part of the regular
sequence and therefore compatible with acquired (implicit or explicit)
sequence knowledge. Third, we allowed participants to integrate
instructed and acquired knowledge during the practice blocks before the
generation task.

Given that the instructed and acquired propositions are identical, we
would argue that qualitative differences between acquired and instructed
knowledge are likely to involve non-propositional forms of knowledge;
such non-propositional knowledge is typically considered to be implicit.
Indeed, it is likely that strong implicit knowledge is a precondition
for acquiring explicit knowledge (Cleeremans \& Jiménez, 2002; Haider \&
Frensch, 2009): Instructed and acquired explicit knowledge are therefore
likely to differ in the degree to which they are correlated with
implicit knowledge. If participants are better able to control acquired
than instructed explicit knowledge, this would then be due,
paradoxically, to the presence of acquired implicit knowledge. Finally,
even if that was the case, note that this would not salvage the PD
method because a strong correlation between explicit and implicit
knowledge would violate the independence assumption, thereby posing
another problem for its validity.

\subsection{Implications}\label{implications}

We will first discuss implications for the PD model and the ordinal
approach before we suggest ways to improve measurement of sequence
knowledge using the generation task. We conclude with a few broader
implications.

\subsubsection{Validity of the PD
method}\label{validity-of-the-pd-method}

The present findings show that participants fail to suppress generating
regular sequences under exclusion instructions. This implies that the
controlled process operates less effectively under exclusion than
inclusion instructions, violating the invariance assumption. A model
that nevertheless incorporates the invariance assumption will likely
fail to adequately account for the data, and will yield distorted
estimates of the automatic and controlled process. To illustrate, assume
that the true values of the parameters are
\(C_{Inclusion} = .8, C_{Exclusion} = .4\), and
\(A_{Inclusion} = A_{Exclusion} = .25\). This yields the following
generation proportions of regular transitions
\(I = .8 + (1-.8)*.25 = 0.85\) and \(E = (1-.4)*.25 = 0.15\). When
fitting a traditional PD model enforcing the invariance assumption
\(C_{Inclusion} = C_{Exclusion}\) to these data, we get \(C=.7\) that
lies somewhere between the true values of \(C\), and \(A = .5\) which is
a vast overestimation of the true \(A\). Importantly, note that if the
true value of \(A=.25\) represents chance level, applications of the
traditional PD method might lead to the erroneous conclusion that
implicit knowledge had been learned even if such knowledge was in fact
entirely absent. In addition, if we are interested in the amount of
explicit knowledge learned from the SRTT training phase, it might be
argued that the higher estimate obtained from the inclusion condition
might be a more valid estimate of learned explicit knowledge; the
inability to express this knowledge under exclusion may be of secondary
interest. By this argument, applying the traditional PD method also
yields an underestimation of explicit knowledge.

We therefore recommend against using the PD method unless separate
estimates of \(C_{Inclusion}\) and \(C_{Exclusion}\) can be obtained,
for example as we have done in the present study. To do so, at least two
levels of an implicit-knowledge factor are necessary across which the
\(C\) parameters could be equated to obtain stable parameter estimates.
Note that this strategy may not be broadly applicable in typical SRTT
studies because of the strong correlation between \(C\) and \(A\); the
assumption that the level of implicit knowledge does not affect the
amount of acquired explicit knowledge will be warranted only in special
cases such as realized in the present studies.\footnote{As another way
  to obtain separate estimates, instead of assuming equal levels of the
  controlled process across levels of the automatic process, one might
  assume that the level of the automatic process does not interact with
  instruction (i.e., it does not affect the relative magnitude of the
  invariance violation). In this case, the controlled parameters need
  not be equated across both levels of implicit knowledge; instead,
  explicit knowledge in the lower level of the implicit-knowledge factor
  can be expressed as a proportion of the explicit knowledge in the
  higher level of that factor (i.e,
  \(C_{Inclusion/low} = w*C_{Inclusion/high}\) and
  \(C_{Exclusion/low} = w*C_{Exclusion/high}\).)}

\subsubsection{Ordinal PD}\label{ordinal-pd}

As argued in the introduction to Experiment 3, it might be the case
that, under certain conditions, certain violations of assumptions
underlying the validity of the PD model equations might leave the
ordinal interpretation of PD data unaffected. This is not true, however,
for the specific violation of invariance of the controlled process
reported here. To reiterate, given a single experimental condition, it
is concluded in the ordinal approach that implicit knowledge is present
if exclusion performance is above a (chance or empirical) baseline; and
it is concluded that explicit knowledge is present if inclusion
performance exceeds exclusion performance. The possible conclusions from
comparing two experimental conditions are outlined by Hirshman (2004,
Table 1). They depend on the invariance assumption in the sense that a
monotonically increasing controlled process should lead to a monotonic
increase of inclusion performance and at the same time a monotonic
decrease of exclusion performance. The present study shows, however,
that exclusion performance cannot be assumed to reliably decrease with
increasing explicit knowledge. This implies that the assumptions
underlying the ordinal-PD approach proposed by Hirshman are also
violated for the generation task as applied to sequence learning. In
addition, we have previously shown that another assumption of ordinal
PD, namely that baseline performance is identical in the inclusion and
exclusion tasks, is also violated at least in some cases (Stahl et al.,
2015). We conclude that the ordinal interpretation of PD data cannot be
recommended as a fallback option.

\subsubsection{Generation task as a measure of sequence
knowledge}\label{generation-task-as-a-measure-of-sequence-knowledge}

The generation task has been proposed as a useful and sensitive measure
of implicit knowledge (Jiménez, Méndez, \& Cleeremans, 1996; Perruchet
\& Amorim, 1992). Its sensitivity may be called into question by the
finding that RT effects obtained during the SRTT were often greater than
implicit-knowledge effects in the generation task. In part, this may be
attributed to the greater reliability of the RT measure, as it relies on
aggregation across a larger number of trials than does the generation
task. Another possible reason is that the generation task's sensitivity
as a measure of implicit knowledge may be lower than previously thought.
For instance, previous findings of implicit knowledge using the
generation task may have been overestimates of implicit knowledge due to
a violation of invariance for the controlled process with \(C_I > C_E\).
Note that most studies used much easier-to-learn materials (with four
instead of six locations); it is thus plausible that participants
acquired more explicit knowledge than they did in our experiments, and
that the overestimation bias was more severe in those studies.

Another possible reason for overestimating implicit knowledge is that
the regularities in the sequences implemented in previous research were
such that the probability of reversals (e.g., 1-2-1) was below chance.
Given that participants spontaneously tend to generate reversals at
below-chance levels, this implies that they instead generate other
regular transitions at slightly above-chance levels even in the absence
of any true sequence knowledge. As a consequence of this
reversal-avoidance bias, implicit knowledge might be overestimated if
one uses chance baselines as a reference. This problem has been
discussed before (Destrebecqz \& Cleeremans, 2003; Reed \& Johnson,
1994; Shanks \& Johnstone, 1999), and was solved by comparing
performance on the training sequence with performance on a transfer
sequence containing a similarly low proportion of reversals. This
implies, however, that the PD approach does not provide a measure of the
absolute level of implicit or explicit knowledge; instead, by relying on
a comparison of performance across two sequences, it yields a difference
measure that is associated with reduced reliability. This is because the
transfer sequence is selected by the experimenter out of a large set of
possible such sequences, and this random choice interacts with
participants' partial acquired knowledge, as well as with their
individual response tendencies, to introduce additional error into the
measurement. In addition, the reversal-avoidance bias may not only mimic
implicit knowledge; it may also mimic (or mask) explicit knowledge if it
interacted with the inclusion-exclusion instructions, perhaps via
different response strategies or criteria adopted under inclusion versus
exclusion instructions.

It might be possible to construct a version of the generation task that
allows for the separation of automatic and controlled processes but does
not depend on exclusion of explicit knowledge and does not induce
different response criteria. For example, D'Angelo, Milliken, Jiménez,
\& Lupiáñez (2013) implemented such a generation task variant in
artificial grammar learning in which two different inclusion
instructions were compared: After learning about two different grammars,
participants were asked, in the first (second) inclusion block to
generate exemplars from the first (second) grammar. Under certain
assumptions, performance differences between blocks can be interpreted
as evidence for explicit controllable knowledge. Exclusion failure and
different criteria presumably do not matter in this task: Participants
were not instructed to exclude explicit knowledge in this task, and it
is plausible that the similarity of instructions for both generation
tasks also induced comparable response criteria. As another example, in
the domain of source memory, the PD procedure can be replaced by a
source-memory task in which, instead of including versus excluding items
from one of two study lists (A and B), participants are asked to
indicate the source of the word (list A or list B; Buchner, Erdfelder,
Steffens, \& Martensen, 1997). Perhaps with a similar modification, an
improved generation task may prove a useful measure of sequence
knowledge.

\subsection{Conclusion}\label{conclusion}

In light of the present findings suggesting limited validity of the PD
generation task, what can we conclude about explicit and implicit
sequence knowledge from its previous applications? We have seen that,
when the traditional PD approach is used, an invariance violation of the
controlled process leads to overestimation of implicit knowledge and to
underestimation of the amount of explicit knowledge participants have
acquired from the preceding SRTT training phase. In addition, an
invariance violation of the automatic process may lead to a (small)
overestimation of explicit knowledge. We can take into account these
potential distortions in our interpretation of previous findings by
distinguishing the two patterns of results found in the literature.

In the first case, evidence for explicit but not for implicit knowledge
was found (e.g., Wilkinson \& Shanks, 2004). In this case, the evidence
for explicit knowledge suggests that the distortions due to the
invariance violation apply. Obtaining evidence for explicit knowledge
despite the underestimation bias implies that explicit knowledge was
likely present. Obtaining no evidence for implicit knowledge despite the
likely presence of an overestimation bias may indeed reflect the absence
of implicit knowledge; alternatively, it may of course reflect lack of
statistical power.

In the second case, evidence for both explicit and implicit knowledge
was reported (e.g., Destrebecqz \& Cleeremans, 2001, 2003; Jiménez et
al., 2006; Mong et al., 2012). Here, two scenarios must be
distinguished: In the first, the evidence for explicit knowledge also
suggests that the distortions resulting from the invariance violation
may have compromised the results: Again, the evidence for explicit
knowledge obtained despite the underestimation bias should probably be
assumed to be reliable; however, the evidence for implicit knowledge may
be an artifact of the overestimation bias and should be interpreted with
caution. In the second scenario, explicit knowledge is absent, and the
explicit-knowledge effect reflects an artifact of the invariance
violation of the automatic process (i.e., \(A_{I} > A_{E}\)); the
results obtained in the literature would then indicate the presence of
implicit but not explicit knowledge.

Of course, different scenarios might underlie different studies, and
single studies may also reflect a mixture of scenarios. Taken together,
when considering the limitations discovered in our studies, the PD
approach to using the generation task as a measure of implicit and
explicit sequence knowledge in the SRTT has so far yielded few reliable
conclusions. Future research should consider using alternative methods
of assessing implicit and explicit knowledge (for a recent overview see
Timmermans \& Cleeremans, 2015).

\subsection{Outlook}\label{outlook}

One of the great benefits of multinomial models such as the PD model is
that they are flexibly adaptable measurement models for studying latent
cognitive processes using a wide variety of experimental paradigms
(Erdfelder et al., 2009). To validate a new model, it is common to
assess its goodness of fit, and to empirically demonstrate that its
parameters can be selectively manipulated and interpreted
psychologically (i.e., parameter estimates reflect targeted experimental
manipulations in the predicted manner; Batchelder \& Riefer, 1999). In
many cases, however, simplifying assumptions need to be made; for
instance, latent processes are equated across two or more experimental
conditions (e.g., a single controlled process \(C\) is assumed to
operate under inclusion and exclusion conditions). Whenever such
assumptions of invariance are made, we propose that they should also be
tested empirically as part of the model-validation effort before a new
model is proposed and used to investigate substantive issues.

\clearpage

\section{References}\label{references}

\setlength{\parindent}{-0.5in} \setlength{\leftskip}{0.5in}
\setlength{\parskip}{8pt}

\hypertarget{refs}{}
\hypertarget{ref-albertux5fbayesianux5f1993}{}
Albert, J. H., \& Chib, S. (1993). Bayesian Analysis of Binary and
Polychotomous Response Data. \emph{Journal of the American Statistical
Association}, \emph{88}(422), 669--679.
doi:\href{https://doi.org/10.1080/01621459.1993.10476321}{10.1080/01621459.1993.10476321}

\hypertarget{ref-batchelderux5ftheoreticalux5f1999}{}
Batchelder, W. H., \& Riefer, D. M. (1999). Theoretical and Empirical
Review of Multinomial Process Tree Modeling. \emph{Psychonomic Bulletin
\& Review}, \emph{6}(1), 57--86.
doi:\href{https://doi.org/10.3758/BF03210812}{10.3758/BF03210812}

\hypertarget{ref-buchnerux5ftowardux5f1995}{}
Buchner, A., Erdfelder, E., \& Vaterrodt-Plünnecke, B. (1995). Toward
Unbiased Measurement of Conscious and Unconscious Memory Processes
within the Process Dissociation Framework. \emph{Journal of Experimental
Psychology: General}, \emph{124}(2), 137--160.
doi:\href{https://doi.org/10.1037/0096-3445.124.2.137}{10.1037/0096-3445.124.2.137}

\hypertarget{ref-buchnerux5fnatureux5f1997}{}
Buchner, A., Erdfelder, E., Steffens, M. C., \& Martensen, H. (1997).
The nature of memory processes underlying recognition judgments in the
process dissociation procedure. \emph{Memory \& Cognition},
\emph{25}(4), 508--517.
doi:\href{https://doi.org/10.3758/BF03201126}{10.3758/BF03201126}

\hypertarget{ref-buchnerux5fmultinomialux5f1997}{}
Buchner, A., Steffens, M. C., Rothkegel, R., \& Erdfelder, E. (1997). A
Multinomial Model to Assess Fluency and Recollection in a Sequence
Learning Task. \emph{The Quarterly Journal of Experimental Psychology
Section A}, \emph{50}(3), 631--663.
doi:\href{https://doi.org/10.1080/713755723}{10.1080/713755723}

\hypertarget{ref-carpenterux5fstanux5finpress}{}
Carpenter, B., Gelman, A., Hoffman, M., Lee, D., Goodrich, B.,
Betancourt, M., \ldots{} Riddell, A. (in press). Stan: A Probabilistic
Programming Language. \emph{Journal of Statistical Software}. Retrieved
from
\url{http://www.stat.columbia.edu/~gelman/research/unpublished/stan-resubmit-JSS1293.pdf}

\hypertarget{ref-cleeremansux5fimplicitux5f2002}{}
Cleeremans, A., \& Jiménez, L. (2002). Implicit Learning and
Consciousness: A Graded, Dynamic Perspective. In R. M. French \& A.
Cleeremans (Eds.), \emph{Implicit Learning and Consciousness: An
Empirical, Philosophical and Computational Consensus in the Making} (pp.
1--40). Hove: Psychology Press. Retrieved from
\url{http://journalpsyche.org/articles/0xc03a.pdf}

\hypertarget{ref-curranux5fimplicitux5f2001}{}
Curran, T. (2001). Implicit Learning Revealed by the Method of
Opposition. \emph{Trends in Cognitive Sciences}, \emph{5}(12), 503--504.
doi:\href{https://doi.org/10.1016/S1364-6613(00)01791-5}{10.1016/S1364-6613(00)01791-5}

\hypertarget{ref-curranux5fviolationsux5f1995}{}
Curran, T., \& Hintzman, D. L. (1995). Violations of the Independence
Assumption in Process Dissociation. \emph{Journal of Experimental
Psychology: Learning, Memory, and Cognition}, \emph{21}(3), 531--547.
doi:\href{https://doi.org/10.1037/0278-7393.21.3.531}{10.1037/0278-7393.21.3.531}

\hypertarget{ref-curranux5fconsequencesux5f1997}{}
Curran, T., \& Hintzman, D. L. (1997). Consequences and Causes of
Correlations in Process Dissociation. \emph{Journal of Experimental
Psychology. Learning, Memory \& Cognition}, \emph{23}(2), 496.
doi:\href{https://doi.org/10.1037/0278-7393.23.2.496}{10.1037/0278-7393.23.2.496}

\hypertarget{ref-destrebecqzux5fcanux5f2001}{}
Destrebecqz, A., \& Cleeremans, A. (2001). Can sequence learning be
implicit? New evidence with the process dissociation procedure.
\emph{Psychonomic Bulletin \& Review}, \emph{8}(2), 343--350.
doi:\href{https://doi.org/10.3758/BF03196171}{10.3758/BF03196171}

\hypertarget{ref-destrebecqzux5ftemporalux5f2003ux5fdoi}{}
Destrebecqz, A., \& Cleeremans, A. (2003). Temporal Effects in Sequence
Learning. In \emph{Attention and Implicit Learning} (pp. 181--213).
Amsterdam, Netherlands: John Benjamins Publishing Company.
doi:\href{https://doi.org/10.1075/aicr.48.11des}{10.1075/aicr.48.11des}

\hypertarget{ref-dangeloux5fimplementingux5f2013}{}
D'Angelo, M. C., Milliken, B., Jiménez, L., \& Lupiáñez, J. (2013).
Implementing Flexibility in Automaticity: Evidence from Context-Specific
Implicit Sequence Learning. \emph{Consciousness and Cognition},
\emph{22}(1), 64--81.
doi:\href{https://doi.org/10.1016/j.concog.2012.11.002}{10.1016/j.concog.2012.11.002}

\hypertarget{ref-erdfelderux5fmultinomialux5f2009}{}
Erdfelder, E., Auer, T.-S., Hilbig, B. E., Aßfalg, A., Moshagen, M., \&
Nadarevic, L. (2009). Multinomial Processing Tree Models: A Review of
the Literature. \emph{Zeitschrift Für Psychologie / Journal of
Psychology}, \emph{217}(3), 108--124.
doi:\href{https://doi.org/10.1027/0044-3409.217.3.108}{10.1027/0044-3409.217.3.108}

\hypertarget{ref-fuux5fcanux5f2010}{}
Fu, Q., Dienes, Z., \& Fu, X. (2010). Can Unconscious Knowledge Allow
Control in Sequence Learning? \emph{Consciousness and Cognition},
\emph{19}(1), 462--474.
doi:\href{https://doi.org/10.1016/j.concog.2009.10.001}{10.1016/j.concog.2009.10.001}

\hypertarget{ref-fuux5fimplicitux5f2008}{}
Fu, Q., Fu, X., \& Dienes, Z. (2008). Implicit Sequence Learning and
Conscious Awareness. \emph{Consciousness and Cognition}, \emph{17}(1),
185--202.
doi:\href{https://doi.org/10.1016/j.concog.2007.01.007}{10.1016/j.concog.2007.01.007}

\hypertarget{ref-haiderux5fconflictsux5f2009}{}
Haider, H., \& Frensch, P. A. (2009). Conflicts between expected and
actually performed behavior lead to verbal report of incidentally
acquired sequential knowledge. \emph{Psychological Research},
\emph{73}(6), 817--834.
doi:\href{https://doi.org/10.1007/s00426-008-0199-6}{10.1007/s00426-008-0199-6}

\hypertarget{ref-haiderux5foldux5f2011}{}
Haider, H., Eichler, A., \& Lange, T. (2011). An Old Problem: How Can We
Distinguish between Conscious and Unconscious Knowledge Acquired in an
Implicit Learning Task? \emph{Consciousness and Cognition},
\emph{20}(3), 658--672.
doi:\href{https://doi.org/10.1016/j.concog.2010.10.021}{10.1016/j.concog.2010.10.021}

\hypertarget{ref-hintzmanux5fmoreux5f1997}{}
Hintzman, D. L., \& Curran, T. (1997). More than One Way to Violate
Independence: Reply to Jacoby and Shrout (1997). \emph{Journal of
Experimental Psychology. Learning, Memory \& Cognition}, \emph{23}(2),
511.
doi:\href{https://doi.org/10.1037/0278-7393.23.2.511}{10.1037/0278-7393.23.2.511}

\hypertarget{ref-hirshmanux5fordinalux5f2004}{}
Hirshman, E. (2004). Ordinal Process Dissociation and the Measurement of
Automatic and Controlled Processes. \emph{Psychological Review},
\emph{111}(2), 553--560.
doi:\href{https://doi.org/10.1037/0033-295X.111.2.553}{10.1037/0033-295X.111.2.553}

\hypertarget{ref-jacobyux5fprocessux5f1991}{}
Jacoby, L. L. (1991). A Process Dissociation Framework: Separating
Automatic from Intentional Uses of Memory. \emph{Journal of Memory and
Language}, \emph{30}(5), 513--541.
doi:\href{https://doi.org/10.1016/0749-596X(91)90025-F}{10.1016/0749-596X(91)90025-F}

\hypertarget{ref-jacobyux5ftowardux5f1997}{}
Jacoby, L. L., \& Shrout, P. E. (1997). Toward a Psychometric Analysis
of Violations of the Independence Assumption in Process Dissociation.
\emph{Journal of Experimental Psychology. Learning, Memory \&
Cognition}, \emph{23}(2), 505.
doi:\href{https://doi.org/10.1037/0278-7393.23.2.505}{10.1037/0278-7393.23.2.505}

\hypertarget{ref-jimenezux5fcomparingux5f1996}{}
Jiménez, L., Méndez, C., \& Cleeremans, A. (1996). Comparing Direct and
Indirect Measures of Sequence Learning. \emph{Journal of Experimental
Psychology: Learning, Memory, and Cognition}, \emph{22}(4), 948.
doi:\href{https://doi.org/10.1037/0278-7393.22.4.948}{10.1037/0278-7393.22.4.948}

\hypertarget{ref-jimenezux5fqualitativeux5f2006}{}
Jiménez, L., Vaquero, J. M., \& Lupiáñez, J. (2006). Qualitative
Differences between Implicit and Explicit Sequence Learning.
\emph{Journal of Experimental Psychology: Learning, Memory, and
Cognition}, \emph{32}(3), 475.
doi:\href{https://doi.org/10.1037/0278-7393.32.3.475}{10.1037/0278-7393.32.3.475}

\hypertarget{ref-klauerux5fhierarchicalux5f2010}{}
Klauer, K. C. (2010). Hierarchical Multinomial Processing Tree Models: A
Latent-Trait Approach. \emph{Psychometrika}, \emph{75}(1), 70--98.
doi:\href{https://doi.org/10.1007/s11336-009-9141-0}{10.1007/s11336-009-9141-0}

\hypertarget{ref-klauerux5finvarianceux5f2015}{}
Klauer, K. C., Dittrich, K., Scholtes, C., \& Voss, A. (2015). The
invariance assumption in process-dissociation models: An evaluation
across three domains. \emph{Journal of Experimental Psychology:
General}, \emph{144}(1), 198--221.
doi:\href{https://doi.org/10.1037/xge0000044}{10.1037/xge0000044}

\hypertarget{ref-lewandowskiux5fgeneratingux5f2009}{}
Lewandowski, D., Kurowicka, D., \& Joe, H. (2009). Generating Random
Correlation Matrices Based on Vines and Extended Onion Method.
\emph{Journal of Multivariate Analysis}, \emph{100}(9), 1989--2001.
doi:\href{https://doi.org/10.1016/j.jmva.2009.04.008}{10.1016/j.jmva.2009.04.008}

\hypertarget{ref-liefoogheux5finstruction-basedux5f2012}{}
Liefooghe, B., Wenke, D., \& De Houwer, J. (2012). Instruction-Based
Task-Rule Congruency Effects. \emph{Journal of Experimental Psychology:
Learning, Memory, and Cognition}, \emph{38}(5), 1325--1335.
doi:\href{https://doi.org/10.1037/a0028148}{10.1037/a0028148}

\hypertarget{ref-mongux5fevidenceux5f2012}{}
Mong, H. M., McCabe, D. P., \& Clegg, B. A. (2012). Evidence of
automatic processing in sequence learning using process-dissociation.
\emph{Advances in Cognitive Psychology / University of Finance and
Management in Warsaw}, \emph{8}(2), 98--108.
doi:\href{https://doi.org/10.2478/v10053-008-0107-z}{10.2478/v10053-008-0107-z}

\hypertarget{ref-nissenux5fattentionalux5f1987}{}
Nissen, M. J., \& Bullemer, P. (1987). Attentional Requirements of
Learning: Evidence from Performance Measures. \emph{Cognitive
Psychology}, \emph{19}(1), 1--32.
doi:\href{https://doi.org/10.1016/0010-0285(87)90002-8}{10.1016/0010-0285(87)90002-8}

\hypertarget{ref-normanux5ffringeux5f2006}{}
Norman, E., Price, M. C., \& Duff, S. C. (2006). Fringe Consciousness in
Sequence Learning: The Influence of Individual Differences.
\emph{Consciousness and Cognition}, \emph{15}(4), 723--760.
doi:\href{https://doi.org/10.1016/j.concog.2005.06.003}{10.1016/j.concog.2005.06.003}

\hypertarget{ref-perruchetux5fconsciousux5f1992}{}
Perruchet, P., \& Amorim, M.-A. (1992). Conscious Knowledge and Changes
in Performance in Sequence Learning: Evidence against Dissociation.
\emph{Journal of Experimental Psychology: Learning, Memory, and
Cognition}, \emph{18}(4), 785.
doi:\href{https://doi.org/10.1037/0278-7393.18.4.785}{10.1037/0278-7393.18.4.785}

\hypertarget{ref-R-base}{}
R Core Team. (2016). \emph{R: A Language and Environment for Statistical
Computing}. Vienna, Austria: R Foundation for Statistical Computing.
Retrieved from \url{https://www.R-project.org/}

\hypertarget{ref-ratcliffux5fprocessux5f1995}{}
Ratcliff, R., Van Zandt, T., \& McKoon, G. (1995). Process Dissociation,
Single-Process Theories, and Recognition Memory. \emph{Journal of
Experimental Psychology: General}, \emph{124}(4), 352--374.
doi:\href{https://doi.org/10.1037/0096-3445.124.4.352}{10.1037/0096-3445.124.4.352}

\hypertarget{ref-reedux5fassessingux5f1994}{}
Reed, J., \& Johnson, P. (1994). Assessing Implicit Learning with
Indirect Tests: Determining What Is Learned about Sequence Structure.
\emph{Journal of Experimental Psychology: Learning, Memory, and
Cognition}, \emph{20}(3), 585--594.
doi:\href{https://doi.org/10.1037/0278-7393.20.3.585}{10.1037/0278-7393.20.3.585}

\hypertarget{ref-rouderux5fintroductionux5f2005}{}
Rouder, J. N., \& Lu, J. (2005). An Introduction to Bayesian
Hierarchical Models with an Application in the Theory of Signal
Detection. \emph{Psychonomic Bulletin \& Review}, \emph{12}(4),
573--604.
doi:\href{https://doi.org/10.3758/BF03196750}{10.3758/BF03196750}

\hypertarget{ref-rouderux5fhierarchicalux5f2008}{}
Rouder, J. N., Lu, J., Morey, R. D., Sun, D., \& Speckman, P. L. (2008).
A Hierarchical Process-Dissociation Model. \emph{Journal of Experimental
Psychology: General}, \emph{137}(2), 370--389.
doi:\href{https://doi.org/10.1037/0096-3445.137.2.370}{10.1037/0096-3445.137.2.370}

\hypertarget{ref-shanksux5fevaluatingux5f1999}{}
Shanks, D. R., \& Johnstone, T. (1999). Evaluating the Relationship
between Explicit and Implicit Knowledge in a Sequential Reaction Time
Task. \emph{Journal of Experimental Psychology: Learning, Memory, and
Cognition}, \emph{25}(6), 1435--1451.
doi:\href{https://doi.org/10.1037/0278-7393.25.6.1435}{10.1037/0278-7393.25.6.1435}

\hypertarget{ref-shanksux5fcharacteristicsux5f1994}{}
Shanks, D. R., \& St. John, M. F. (1994). Characteristics of Dissociable
Human Learning Systems. \emph{Behavioral and Brain Sciences},
\emph{17}(3), 367--395.
doi:\href{https://doi.org/10.1017/S0140525X00035032}{10.1017/S0140525X00035032}

\hypertarget{ref-shanksux5fattentionalux5f2005}{}
Shanks, D. R., Rowland, L. A., \& Ranger, M. S. (2005). Attentional load
and implicit sequence learning. \emph{Psychological Research},
\emph{69}(5-6), 369--382.
doi:\href{https://doi.org/10.1007/s00426-004-0211-8}{10.1007/s00426-004-0211-8}

\hypertarget{ref-spiegelhalterux5fbayesianux5f2002}{}
Spiegelhalter, D. J., Best, N. G., Carlin, B. P., \& Van Der Linde, A.
(2002). Bayesian Measures of Model Complexity and Fit. \emph{Journal of
the Royal Statistical Society: Series B (Statistical Methodology)},
\emph{64}(4), 583--639.
doi:\href{https://doi.org/10.1111/1467-9868.00353}{10.1111/1467-9868.00353}

\hypertarget{ref-stahlux5fdistortedux5f2015}{}
Stahl, C., Barth, M., \& Haider, H. (2015). Distorted Estimates of
Implicit and Explicit Learning in Applications of the
Process-Dissociation Procedure to the SRT Task. \emph{Consciousness and
Cognition}, \emph{37}, 27--43.
doi:\href{https://doi.org/10.1016/j.concog.2015.08.003}{10.1016/j.concog.2015.08.003}

\hypertarget{ref-timmermansux5fhowux5f2015}{}
Timmermans, B., \& Cleeremans, A. (2015). How Can We Measure Awareness?
An Overview of Current Methods. In \emph{Behavioural Methods in
Consciousness Research} (pp. 21--46). Oxford: Oxford University Press.

\hypertarget{ref-wilkinsonux5fintentionalux5f2004}{}
Wilkinson, L., \& Shanks, D. R. (2004). Intentional Control and Implicit
Sequence Learning. \emph{Journal of Experimental Psychology: Learning,
Memory, and Cognition}, \emph{30}(2), 354--369.
doi:\href{https://doi.org/10.1037/0278-7393.30.2.354}{10.1037/0278-7393.30.2.354}

\hypertarget{ref-yonelinasux5fprocess-dissociationux5f2012}{}
Yonelinas, A. P., \& Jacoby, L. L. (2012). The process-dissociation
approach two decades later: Convergence, boundary conditions, and new
directions. \emph{Memory \& Cognition}, \emph{40}(5), 663--680.
doi:\href{https://doi.org/10.3758/s13421-012-0205-5}{10.3758/s13421-012-0205-5}




  \begin{appendix}
  \section{}
  This appendix provides a complete specification of the models and priors
  used. Code (R/rstan) is available at
  \url{https://github.com/methexp/pdl2}.
  
  \subsection{\texorpdfstring{Experiment 1, model
  \(\mathcal{M}_1\)}{Experiment 1, model \textbackslash{}mathcal\{M\}\_1}}\label{experiment-1-model-mathcalmux5f1}
  
  Priors on fixed effects were
  
  \[
  \begin{aligned}
  \mu_{km}^{(C)} \sim & N(0, 1), k = \lbrace 1, 2 \rbrace; m = \lbrace 1, 2 \rbrace\\
  \mu_{jkm}^{(A)} \sim & N(0, 1), j = \lbrace 1, 2 \rbrace; k = \lbrace 1, 2 \rbrace; m = \lbrace 1, 2 \rbrace
  \end{aligned}
  \]
  
  where \(j\) indexes transition type (revealed vs.~non-revealed), \(k\)
  indexes learning material presented during the SRTT (random
  vs.~probabilistic), and \(m\) indexes \emph{PD instruction} condition
  (inclusion vs.~exclusion).
  
  For participants who did not receive explicit knowledge about a single
  transition, we assumed that all \(C_{ijm} = 0\). Therefore, participant
  effects are only required for automatic processes
  (\(\delta_{ijm}^{(A)}\)). In participants who received explicit
  knowledge about one transition, two additional participant effects were
  needed to model controlled processes for revealed transitions
  (\(\delta_{im}^{(C)}\)). We thus provide the specification of
  participant effects for these two groups of participants separately.
  
  \paragraph{Participants who did not receive explicit knowledge about one
  transition}\label{participants-who-did-not-receive-explicit-knowledge-about-one-transition}
  
  For participants who did not receive explicit knowledge about one
  transition, participant effects \(\delta_{ijm}^{(A)}\) can be written as
  vectors \(\boldsymbol{\delta}_i\) that were modeled as draws from a
  multivariate normal
  
  \[
  \boldsymbol{\delta}_i \sim N_4 (0, \Sigma_{kl}), i = 1, ..., I
  \]
  
  where \(k\) indexes the learning material that was presented to
  participant \(i\) and \(l\) indexes his or her level of the
  explicit-knowledge factor. The covariance matrices \(\Sigma_{kl}\) were
  obtained from the standard deviations of participant effects
  \(\boldsymbol{\sigma}_{kl}\) and correlation matrices \(\Omega_{kl}\)
  
  \[
  \Sigma_{kl} = Diag(\boldsymbol{\sigma}_{kl})~\Omega_{kl}~Diag(\boldsymbol{\sigma}_{kl}), k = \lbrace 1, 2 \rbrace, l = \lbrace 1, 2 \rbrace
  \]
  
  Each element \(\sigma_{klp}\) of the vectors of standard deviations
  \(\boldsymbol{\sigma}_{kl}\) was drawn from independent half-normal
  prior distributions.
  
  \[
  \sigma_{klp} \sim N (0, 1)_{\mathcal{I}(0, \infty)}, k = \lbrace 1, 2 \rbrace, l = \lbrace 1, 2 \rbrace
  \]
  
  For the correlation matrices \(\Omega_{k}\), we used LKJ priors with a
  scaling factor of 1 (Lewandowski, Kurowicka, \& Joe, 2009):
  
  \[
  \Omega_{kl} \sim \textit{LKJcorr}(\nu = 1), k = \lbrace 1, 2 \rbrace, l = \lbrace 1, 2 \rbrace
  \]
  
  \paragraph{Participants who received explicit knowledge about one
  transition}\label{participants-who-received-explicit-knowledge-about-one-transition}
  
  For participants who received explicit knowledge about one transition,
  participant effects \(\delta_{ijm}^{(A)}\) and \(\delta_{im}^{(C)}\) can
  be written as vectors \(\boldsymbol{\delta}_i\) that were modeled as
  draws from a multivariate normal
  
  \[
  \boldsymbol{\delta}_i \sim N_6 (0, \Sigma_{kl}), i = 1, ..., I
  \]
  
  where \(k\) indexes the learning material that was presented to
  participant \(i\) and \(l\) indexes his or her level of the
  explicit-knowledge factor. The covariance matrices \(\Sigma_{kl}\) were
  specified as above, with the only exception that six instead of four
  parameters were required.
  
  \subsection{\texorpdfstring{Experiment 1, model
  \(\mathcal{M}_2\)}{Experiment 1, model \textbackslash{}mathcal\{M\}\_2}}\label{experiment-1-model-mathcalmux5f2}
  
  Priors on fixed effects were
  
  \[
  \begin{aligned}
  \mu_{jkl}^{(C)} \sim & N(0, 1), j = \lbrace 1, 2 \rbrace; k = \lbrace 1, 2 \rbrace; l = \lbrace 1, 2 \rbrace\\
  \mu_{jkl}^{(A)} \sim & N(0, 1), j = \lbrace 1, 2 \rbrace; k = \lbrace 1, 2 \rbrace; l = \lbrace 1, 2 \rbrace\\
  \end{aligned}
  \]
  
  Participant effects \(\delta_{ij}^{(A)}\) and \(\delta_{ij}^{(C)}\) can
  be written as vectors \(\boldsymbol{\delta}_i\) that were modeled by \[
  \boldsymbol{\delta}_i \sim N_4 (0, \Sigma_{kl}), i = 1, ..., I
  \] Priors for the covariance matrix \(\Sigma_{kl}\) were specified as
  above.
  
  \subsection{\texorpdfstring{Experiment 2, models \(\mathcal{M}_1\) and
  \(\mathcal{M}_2\)}{Experiment 2, models \textbackslash{}mathcal\{M\}\_1 and \textbackslash{}mathcal\{M\}\_2}}\label{experiment-2-models-mathcalmux5f1-and-mathcalmux5f2}
  
  For the model-based analyses, we used models \(\mathcal{M}_1\) and
  \(\mathcal{M}_2\) analogous to those used in Experiment 1.
  
  \subsection{\texorpdfstring{Experiment 3, model
  \(\mathcal{M}_1\)}{Experiment 3, model \textbackslash{}mathcal\{M\}\_1}}\label{experiment-3-model-mathcalmux5f1}
  
  Priors on fixed effects were
  
  \[
  \begin{aligned}
  \mu_{jlm}^{(C)} & \sim N(0, 1), j = \lbrace 1, 2 \rbrace; l = \lbrace 1, 2 \rbrace; m = \lbrace 1, 2 \rbrace\\
  \mu_{mt}^{(A)} & \sim N(0, 1), t = \lbrace 1, ..., 6 \rbrace ; m = \lbrace 1, 2 \rbrace\\
  \end{aligned}
  \]
  
  where \(j\) indexes \emph{transition type} (revealed \& practiced
  vs.~revealed \& non-practiced), \(l\) indexes practice condition
  (Control, No-practice, Unspecific-practice, Practice, Transfer), \(t\)
  indexes specific items (i.e., transitions), and \(m\) indexes \emph{PD
  instruction} (inclusion vs.~exclusion).
  
  Participant effects \(\delta_{imt}^{(A)}\) and \(\delta_{ijm}^{(C)}\)
  can be written as vectors \(\boldsymbol{\delta}_i\).
  
  For participants in the \emph{Control} group, these were modeled by \[
  \boldsymbol{\delta}_i \sim N_{12} (0, \Sigma_l), i = 1, ..., I
  \]
  
  For participants in the \emph{No-Practice}, \emph{Unspecific-Practice},
  and \emph{Practice} groups, \[
  \boldsymbol{\delta}_i \sim N_{14} (0, \Sigma_l), i = 1, ..., I
  \]
  
  For participants in the \emph{Transfer} group \[
  \boldsymbol{\delta}_i \sim N_{16} (0, \Sigma_l), i = 1, ..., I
  \]
  
  The covariance matrices \(\Sigma_l\) were modeled separately and
  independently for each between-subjects condition. Priors on these
  matrices were as described above for Experiment 1.
  
  \hypertarget{refs}{}
  \end{appendix}
  
% Figures

\clearpage

\begin{figure}[htbp]
\centering
\includegraphics{main_files/figure-latex/exp1_acquisition_RT-1.pdf}
\caption{RTs during acquisition phase, split by \emph{material} and
\emph{FOC transition status}. Error bars represent 95\% within-subjects
confidence intervals.}
\end{figure}

\begin{figure}[htbp]
\centering
\includegraphics{main_files/figure-latex/exp1_acquisition_err-1.pdf}
\caption{Error rates during acquisition phase, split by \emph{material}
and \emph{FOC transition status}. Error bars represent 95\%
within-subjects confidence intervals.}
\end{figure}

\begin{figure}[htbp]
\centering
\includegraphics{main_files/figure-latex/unnamed-chunk-6-1.pdf}
\caption{Parameter estimates from Experiment 1. Error bars represent
95\% confidence intervals.}
\end{figure}

\begin{figure}[htbp]
\centering
\includegraphics{main_files/figure-latex/figure_exp1-1.pdf}
\caption{Posterior differences between \(A_I - A_E\) and \(C_I - C_E\)
in Experiment 1, plotted for each participant (gray dots) with 95\%
credible intervals. Dashed lines represent the posterior means of the
differences between mean parameter estimates. Dotted lines represent
95\% credible intervals.}
\end{figure}

\begin{figure}[htbp]
\centering
\includegraphics{main_files/figure-latex/exp2_acq_RT-1.pdf}
\caption{RTs during acquisition phase, split by \emph{material} and
\emph{SOC transition status}. Error bars represent 95\% within-subjects
confidence intervals.}
\end{figure}

\begin{figure}[htbp]
\centering
\includegraphics{main_files/figure-latex/exp2_acq_err-1.pdf}
\caption{Error rates during acquisition phase, split by \emph{material}
and \emph{SOC transition status}. Error bars represent 95\%
within-subjects confidence intervals.}
\end{figure}

\begin{figure}[htbp]
\centering
\includegraphics{main_files/figure-latex/unnamed-chunk-7-1.pdf}
\caption{Parameter estimates from Experiment 2. Error bars represent
95\% confidence intervals.}
\end{figure}

\begin{figure}[htbp]
\centering
\includegraphics{main_files/figure-latex/unnamed-chunk-8-1.pdf}
\caption{Posterior differences \(A_I - A_E\) and \(C_I - C_E\) in
Experiment 2, plotted for each participant (gray dots) with 95\%
credible intervals. Dashed lines represent the posterior means of the
differences between mean parameter estimates. Dotted lines represent
95\% credible intervals.}
\end{figure}

\begin{figure}[htbp]
\centering
\includegraphics{main_files/figure-latex/exp3_acquisition_RT-1.pdf}
\caption{RTs during acquisition phase, split by \emph{FOC transition
status}. Error bars represent 95\% within-subjects confidence
intervals.\label{fig:figure_exp3_RT}}
\end{figure}

\begin{figure}[htbp]
\centering
\includegraphics{main_files/figure-latex/exp3_acquisition_err-1.pdf}
\caption{Error rates during acquisition phase, split by \emph{FOC
transition status}. Error bars represent 95\% within-subjects confidence
intervals. \label{fig:figure_exp3_ER}}
\end{figure}

\begin{figure}[htbp]
\centering
\includegraphics{main_files/figure-latex/unnamed-chunk-11-1.pdf}
\caption{Parameter estimates from Experiment 3. Error bars represent
95\% confidence intervals.}
\end{figure}

\begin{figure}[htbp]
\centering
\includegraphics{main_files/figure-latex/unnamed-chunk-12-1.pdf}
\caption{Posterior differences between \(A_I - A_E\) and \(C_I - C_E\)
in Experiment 3, plotted for each participant (gray dots) with 95\%
credible intervals. Dashed lines represent the posterior means of the
differences between mean parameter estimates. Dotted lines represent
95\% credible intervals.}
\end{figure}

\begin{figure}[htbp]
\centering
\includegraphics{main_files/figure-latex/exp3_gen_revealed-1.pdf}
\caption{Generation performance for revealed transitions. Error bars
represent 95\% confidence intervals. Horizontal lines represent chance
baseline.}
\end{figure}

\begin{figure}[htbp]
\centering
\includegraphics{main_files/figure-latex/exp3_gen_revealed_hyps_incl-1.pdf}
\caption{Inclusion performance for revealed transitions. Left:
Between-subjects comparison between \emph{No-Practice} and
\emph{Practice} groups. Right: Within-subjects comparison in
\emph{Transfer} group. Horizontal lines represent chance baseline.}
\end{figure}

\begin{figure}[htbp]
\centering
\includegraphics{main_files/figure-latex/exp3_gen_revealed_hyps-1.pdf}
\caption{Exclusion performance for revealed transitions. Left:
Between-subjects comparison between \emph{No-Practice} and
\emph{Practice} groups. Right: Within-subjects comparison in
\emph{Transfer} group. Horizontal lines represent chance baseline.}
\end{figure}

\end{document}
